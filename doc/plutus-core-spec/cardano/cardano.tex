\chapter{Plutus Core on Cardano}
\section{Protocol versions}
The Cardano blockchain controls the introduction of features through the use of \emph{protocol versions}, a field in the protocol parameters.
The major protocol version is used to indicate when forwards-incompatible changes (i.e. those that allow blocks that were not previously allowed) are made to the rules of the chain.
This is a hard fork of the chain.

The protocol version is part of the history of the chain, as are all protocol parameters.
That means that all blocks are associated with the protocol version from when they were created, so that they can be interpreted correctly.

In summary, conditioning on the protocol version is the main way in which we can introduce changes in behaviour.

Table~\ref{table:protocol-versions} lists the protocol versions that are relevant to the use of Plutus Core on Cardano.

\begin{table}[H]
  \centering
    \begin{tabular}{|c|l|l|}
        \hline
        \thead{Protocol version} & \thead{Codename} & \thead{Date} \\
        \hline
        5.0 & Alonzo & September 2021 \\
        7.0 & Vasil & June 2022 \\
        8.0 & Valentine & February 2023 \\
        9.0 & Conway & Upcoming \\
        \hline
    \end{tabular}
    \caption{Protocol versions}
    \label{table:protocol-versions}
\end{table}

\section{Ledger languages}

The Cardano ledger uses Plutus Core as the programming language for \emph{scripts}.
The ledger in fact supports multiple different interpretations for scripts, and so each script is tagged with a \emph{ledger language} that tells the ledger how to interpret it.
Since the ledger must always be able to evaluate old scripts and get the same answer, the ledger language must pin down everything about how the script is evaluated, including:

\begin{enumerate}
  \item How to interpret the script itself (e.g. as a Plutus Core program, what versions of the Plutus Core language are allowable)
  \item Other configuration the script may need in order to run (e.g. the set of builtin types and functions and their interpretations, cost model parameters)
  \item How the script is invoked (e.g. after having certain arguments passed to it)
\end{enumerate}

There are currently three ``Plutus'' ledger languages (i.e. ledger languages whose underlying programming language is Plutus Core) in use on Cardano:\footnote{
  Note that ledger languages are completely distinct from the point of view of the ledger, the ``V1''/``V2'' naming is suggestive of the fact that these two ledger languages are related, but in the implementation they are completely independent.
}
\begin{enumerate}
  \item \LL{PlutusV1}
  \item \LL{PlutusV2}
  \item \LL{PlutusV3} (forthcoming)
\end{enumerate}

\noindent Table~\ref{table:ll-introduction} shows when each Plutus ledger language was introduced.
Ledger languages remain available permanently after they have been introduced.

\begin{table}[H]
  \centering
    \begin{tabular}{|c|c|}
        \hline
        \thead{Protocol version} & \thead{Ledger language introduced} \\
        \hline
        5.0 & \LL{PlutusV1} \\
        7.0 & \LL{PlutusV2} \\
        9.0 & \LL{PlutusV3} \\
        \hline
    \end{tabular}
    \caption{Introduction of Plutus ledger languages}
    \label{table:ll-introduction}
\end{table}

\noindent Ledger languages can evolve over time.
We can make backwards-compatible changes when the major protocol version changes, but backwards-incompatible changes can only be introduced by creating a whole new ledger language.\footnote{
  See~\cite{CIP-35} for more details on how the process of evolution works.
}
This means that to fully explain the behaviour of a ledger language we may need to also index by the protocol version.

The following tables show how Plutus ledger languages determine:
\begin{itemize}
  \item Which Plutus Core language versions are allowable (Table~\ref{table:lv-introduction})
  \item Which built-in functions and types are available (Table~\ref{table:b-introduction}, given in terms of batches, see Section~\ref{sec:builtin-batches})
  \item How to interpret the built-in functions and types (Table~\ref{table:bs-introduction}, given in terms of built-in semantics variants, see Section~\ref{sec:builtin-semantics-variants})
\end{itemize}

Currently, once we add a feature for any given protocol version/ledger language, we also make it available for all subsequent protocol versions/ledger languages.
For example, Batch 2 of builtins was introduced in \LL{PlutusV2} at protool version 7.0, so it is also available in \LL{PlutusV2} at protocol versions after 7.0, and \LL{PlutusV3} at protocol versions after 9.0 (when \LL{PlutusV3} itself is first introduced).
Hence the tables are simplified to only show when something is \emph{introduced}.

\begin{table}[H]
  \centering
    \begin{tabular}{|c|c|c|}
        \hline
        \thead{Ledger language} & \thead{Protocol version} & \thead{Plutus Core language version introduced} \\
        \hline
        \LL{PlutusV1} & 5.0 & 1.0.0 \\
        \LL{PlutusV3} & 9.0 & 1.1.0 \\
        \hline
    \end{tabular}
    \caption{Introduction of Plutus Core language versions}
    \label{table:lv-introduction}
\end{table}

\begin{table}[H]
  \centering
    \begin{tabular}{|c|c|c|}
        \hline
        \thead{Ledger language} & \thead{Protocol version} & \thead{Built-in functions and types introduced} \\
        \hline
        \LL{PlutusV1} & 5.0 & Batch 1 \\
        \LL{PlutusV2} & 7.0 & Batch 2 \\
        \LL{PlutusV2} & 8.0 & Batch 3 \\
        \LL{PlutusV3} & 9.0 & Batch 4 \\
        \hline
    \end{tabular}
    \caption{Introduction of built-in functions and types}
    \label{table:b-introduction}
\end{table}

\begin{table}[H]
  \centering
    \begin{tabular}{|c|c|c|}
        \hline
        \thead{Ledger language} & \thead{Built-in semantics variant used} \\
        \hline
        \LL{PlutusV1} & Built-in semantics 1 \\
        \LL{PlutusV3} & Built-in semantics 2 \\
        \hline
    \end{tabular}
    \caption{Selection of built-in semantics variant}
    \label{table:bs-introduction}
\end{table}

\section{Built-in types and functions}
\label{sec:cardano-builtins}
\paragraph{Built-in batches.}
\label{sec:builtin-batches}

The built-in types and functions are defined in batches corresponding to how
they were added to ledger languages.  These batches are given in the following
sections.

\paragraph{Built-in semantics variants.}
\label{sec:builtin-semantics-variants}

In rare cases we can make a mistake or need to change the actual behaviour of a
built-in function.  To handle this we define a series of built-in semantics
variants, which indicate which behaviour should be used.  A fix will typically
be deployed by defining a new semantics variant, and then using that variant for
future ledger languages (but not existing ones, since this is usually a
backwards-incompatible change).

Changes are listed alongside the original definition of the built-in function in
its original batch, and are indexed in the following table.

\begin{table}[H]
  \centering
    \begin{tabular}{|l|l|l|}
        \hline
        \thead{Built-in semantics variant} & \thead{Changes from previous semantics} \\
        \hline
        Built-in semantics 1 & None \\
        Built-in semantics 2 & \TT{consByteString} (See~\ref{note:consbytestring}) \\
        \hline
    \end{tabular}
    \caption{Built-in semantics variants}
    \label{table:bs-variants}
\end{table}


\paragraph{Concrete syntax for built-in types.}
Recall that in the abstract notation for built-in types introduced in
Section~\ref{sec:built-in-types}, a built-in type is either an \textit{atomic
  type} such as \texttt{integer} or \texttt{string} or an application $\op(T_1,
\ldots, T_n)$ of a \textit{type operator} to a sequence of built-in types.  The
concrete syntax of built-in types used in textual Plutus Core programs is
slightly different in that we use a curried form of application for type
operators: a type is given by

\newcommand{\bitn}{\mathbf{T}} % built-in type name
\newcommand{\bitc}{\mathbf{C}} % built-in constant syntax

\begin{minipage}{\linewidth}
    \centering
    \[\begin{array}{rclr}
    \bitn & ::= & \textit{atomic-type} & \textrm{Atomic type}\\
                & & \texttt{(}\op \ \bitn_1 \ldots \bitn_{\valency{\op}}) & \textrm{Type application}\\
    \end{array}\]
    \label{fig:built-in-type-concrete-syntax}
\end{minipage}
\noindent Note that we again require that all type operators are fully applied.
We refer to the syntactic objects $\bitn$ above as \textit{concrete built-in
 types}. There is an obvious bijection between these and the abstract built-in
types used elsewhere in this document, and given an abstract built-in type $T$
we will denote the corresponding concrete built-in type by $\bar{T}$.

\paragraph{Concrete syntax for built-in constants.}
We provide concrete syntax for constants of most (but not all) built-in types.
For a built-in type $T$ which has a concrete syntax we specify a set $\bitc_T$
of strings (using either regular expressions or a BNF-style grammar), and a
constant of type $T$ is then represented in the concrete syntax by an expression
of the form \texttt{(con $\bar{T}$ $c_T$)} with $c_T \in \bitc_T$.  Each string
$c_T$ will have an interpretation as a value of type $T$ (ie, an element of
$\denote{T}$) and since this will generally be the obvious interpretation we
will not always spell out the details.%%
\nomenclature[H]{$\bitc_T$}{Set of strings used for the concrete syntax of constants of built-in type $T$}


\newcounter{notenumberA}
\newcommand{\note}[1]{
  \bigskip
  \refstepcounter{notenumberA}
  \noindent\textbf{Note \thenotenumberA. #1}
}

\newcommand{\utfeight}{\mathsf{utf8}}
\newcommand{\unutfeight}{\mathsf{utf8}^{-1}}
\newcommand{\vk}{\textit{vk}}  %% Verification key (ie public key) for signature verification functions.

\subsection{Batch 1}
\label{sec:default-builtins-1}

\subsubsection{Built-in types and type operators}
\label{sec:built-in-types-1}
The first batch of builtin types and type operators is defined in Tables~\ref{table:built-in-types-1}
and~\ref{table:built-in-type-operators-1}.  We also include concrete syntax for
these; the concrete syntax is not strictly part of the language, but may be
useful for tools working with Plutus Core.

\begin{table}[H]
  \centering
    \begin{tabular}{|l|p{6cm}|l|}
        \hline
        Type & Denotation & Concrete Syntax\\
        \hline
        \texttt{integer} &   $\mathbb{Z}$ & \texttt{-?[0-9]+}\\
        \texttt{bytestring}  & $ \B^*$, the set of sequences of bytes or 8-bit characters. & \texttt{\#([0-9A-Fa-f][0-9A-Fa-f])*}\\
        \texttt{string} & $\U^*$,  the set of sequences of Unicode characters. & See note below\\
        \texttt{bool} & \{\texttt{true, false}\} & \texttt{True | False}\\
        \texttt{unit} &  \{()\} & \texttt{()}\\
        \texttt{data} &  See below & See below\\
        \hline
    \end{tabular}
    \caption{Atomic Types}
    \label{table:built-in-types-1}
\end{table}

\begin{table}[H]
  \centering
    \begin{tabular}{|l|p{14mm}|l|l|}
        \hline
        Operator $\op$ & $\left|\op\right|$  & Denotation & Concrete Syntax\\
        \hline
        \texttt{list} & 1 & $\denote{\listOf{t}} = \denote{t}^*$ & See below\\
        \texttt{pair} & 2 & $\denote{\pairOf{t_1}{t_2}} = \denote{t_1} \times \denote{t_2}$ & See below\\
        \hline
        \end{tabular}
   \caption{Type Operators}
    \label{table:built-in-type-operators-1}
\end{table}

\paragraph{Concrete syntax for strings.} Strings are represented as sequences of Unicode characters
enclosed in double quotes, and may include standard escape sequences.  Surrogate
characters in the range \texttt{U+D800}--\texttt{U+DFFF} are replaced with the
Unicode replacement character \texttt{U+FFFD}.


\paragraph{Concrete syntax for lists and pairs.}
A list of type $\texttt{list}(t)$ is written as a syntactic list
\texttt{[$c_1, \ldots, c_n$]} where each $c_i$ lies in $\bitc_t$; a pair of type
$\texttt{pair}(t_1,t_2)$ is written as a syntactic pair $\texttt{(}c_1,c_2\texttt{)}$
with $c_1 \in \bitc_{t_1}$ and $c_2 \in \bitc_{t_2}$.  Some valid constant expressions
are thus

\begin{verbatim}
   (con (list integer) [11, 22, 33])
   (con (pair bool string) (True, "Plutus")).
   (con (list (pair bool (list bytestring)))
      [(True, []), (False, [#,#1F]), (True, [#123456, #AB, #ef2804])])
\end{verbatim}


\paragraph{The $\ty{data}$ type.}
We provide a built-in type $\ty{data}$ which permits the encoding of simple data
structures for use as arguments to Plutus Core scripts.  This type is defined in
Haskell as
\begin{alltt}
   data Data =
      Constr Integer [Data]
      | Map [(Data, Data)]
      | List [Data]
      | I Integer
      | B ByteString
\end{alltt}

\noindent In set-theoretic terms the denotation of $\ty{data}$ is
defined to be the least fixed point of the endofunctor $F$ on the category of
sets given by $F(X) = (\denote{\ty{integer}} \times X^*) \disj (X \times X)^* \disj
X^* \disj \denote{\ty{integer}} \disj \denote{\ty{bytestring}}$, so that
$$ \denote{\ty{data}} = (\denote{\ty{integer}} \times \denote{\ty{data}}^*)
               \disj (\denote{\ty{data}} \times \denote{\ty{data}})^*
               \disj \denote{\ty{data}}^*
               \disj \denote{\ty{integer}}
               \disj \denote{\ty{bytestring}}.
$$
We have injections
\begin{align*}
  \inj_C: \denote{\ty{integer}} \times \denote{\ty{data}}^* & \to \denote{\ty{data}} \\
  \inj_M: \denote{\ty{data}} \times \denote{\ty{data}}^*  & \to \denote{\ty{data}} \\
  \inj_L: \denote{\ty{data}}^* & \to \denote{\ty{data}} \\
  \inj_I: \denote{\ty{integer}} & \to \denote{\ty{data}} \\
  inj_B: \denote{\ty{bytestring}} & \to \denote{\ty{data}} \\
\end{align*}
\noindent and projections
\begin{align*}
  \proj_C: \denote{\ty{data}} & \to \withError{(\denote{\ty{integer}} \times \denote{\ty{data}}^*)}\\
  \proj_M: \denote{\ty{data}} & \to \withError{(\denote{\ty{data}} \times \denote{\ty{data}}^*)}\\
  \proj_L: \denote{\ty{data}} & \to \withError{\denote{\ty{data}}^* }\\
  \proj_I: \denote{\ty{data}} & \to \withError{\denote{\ty{integer}}}\\
  \proj_B: \denote{\ty{data}} & \to \withError{\denote{\ty{bytestring}} }\\
\end{align*}
\noindent which extract an object of the relevant type from a $\ty{data}$ object
$D$, returning $\errorX$ if $D$ does not lie in the expected component of the
disjoint union; also there are functions
$$
\is_C, \is_M, \is_L, \is_I, \is_B: \denote{\ty{data}} \to \denote{\ty{bool}}
$$
\noindent which determine whether a $\ty{data}$ value lies in the relevant component.

\paragraph{Note: \texttt{Constr} tag values.}
\label{note:constr-tag-values}
The \texttt{Constr} constructor of the \texttt{data} type is intended to
represent values from algebraic data types (also known as sum types and
discriminated unions, among other things; \texttt{data} itself is an example of
such a type), where $\mathtt{Constr}\, i\, [d_1,\ldots,d_n]$
represents a tuple of data items together with a tag $i$ indicating which of a
number of alternatives the data belongs to.  The definition above allows tags to
be any integer value, but because of restrictions in the serialisation format
for \texttt{data} (see Section~\ref{sec:encoding-data}) we recommend that in
practice \textbf{only tags $i$ with $0 \leq i \leq 2^{64}-1$ should be used}:
deserialisation will fail for \texttt{data} items (and programs which include
such items) involving tags outside this range.

Note also that \texttt{Constr} is unrelated to the $\keyword{constr}$ term in
Plutus Core itself. Both provide ways of representing structured data, but
the former is part of a built-in type whereas the latter is part of the language
itself.

\newcommand{\syn}[1]{c_{\mathtt{{#1}}}}

\paragraph{Concrete syntax for $\ty{data}$.}
The concrete syntax for $\ty{data}$ is given by

\begin{minipage}{0.6\linewidth}
    \centering
    \[\begin{array}{rcl}
    \syn{data} & ::= & \texttt{(Constr} \ \syn{integer} \ \syn{list(data)} \texttt{)}\\
               &  & \texttt{(Map} \ \syn{list(pair(data,data))} \texttt{)}\\
               &  & \texttt{(List} \ \syn{list(data)} \texttt{)}\\
               &  & \texttt{(I} \ \syn{integer} \texttt{)}\\
               &  & \texttt{(B} \ \syn{bytestring} \texttt{)}.
    \end{array}\]
    \label{fig:data-concrete-syntax}
\end{minipage}

\noindent We interpret these syntactic constants as elements of $\denote{\ty{data}}$ using
the various `$\inj$' functions defined earlier.  Some valid \texttt{data}
constants are

\begin{verbatim}
   (con data (Constr 1 [(I 2), (B #), (Map [])]) 
   (con data (Map [((I 0), (B #00)), ((I 1), (B #0F))])) 
   (con data (List [(I 0), (I 1), (B #7FFF), (List []]))) 
   (con data (I -22)) 
   (con data (B #001A)).
\end{verbatim}
% TODO: be more relaxed about parenthesisation in general

\paragraph{Note.}  At the time of writing the syntax accepted by IOG's parser for textual Plutus Core
differs slightly from that above in that subobjects
of \texttt{Constr}, \texttt{Map} and \texttt{List} objects must \textit{not} be
parenthesised: for example one must write \verb|(con data (Constr 1 [I 2, B #,Map []])|.
This discrepancy will be resolved in the near future.


\subsubsection{Built-in functions}
\label{sec:built-in-functions-1}
The first batch of built-in functions is shown in
Table~\ref{table:built-in-functions-1}.  The table indicates which functions can
fail during execution, and conditions causing failure are specified either in
the denotation given in the table or in a relevant note.  Recall also that a
built-in function will fail if it is given an argument of the wrong type: this
is checked in conditions involving the $\sim$ relation and the $\Eval$ function
in Figures~\ref{fig:untyped-term-reduction} and~\ref{fig:untyped-cek-machine}.
Note also that some of the functions are
\#-polymorphic.  According to Section~\ref{sec:builtin-denotations} we
require a denotation for every possible monomorphisation of these; however all
of these functions are parametrically polymorphic so to simplify notation we
have given a single denotation for each of them with an implicit assumption that
it applies at each possible monomorphisation in an obvious way.

\setlength{\LTleft}{-18mm} % Shift the table left a bit to centre it on the page
\begin{longtable}[H]{|l|p{5cm}|p{5.5cm}|c|c|}
    \hline
    \text{Function} & \text{Signature} & \text{Denotation} & \text{Can} & \text{Note} \\
    & & & fail? & \\
    \hline
    \endfirsthead
    \hline
    \text{Function} & \text{Type} & \text{Denotation} & \text{Can} & \text{Note}\\
    & & & fail? & \\
    \hline
    \endhead
    \hline
    \caption{Built-in Functions}
    % This caption goes on every page of the table except the last.  Ideally it
    % would appear only on the first page and all the rest would say
    % (continued). Unfortunately it doesn't seem to be easy to do that in a
    % longtable.
    \endfoot
    \caption[]{Built-in Functions (continued)}
    \label{table:built-in-functions-1}
    \endlastfoot
    \TT{addInteger}               & $[\ty{integer}, \ty{integer}] \to \ty{integer}$   & $+$ &  & \\
    \TT{subtractInteger}          & $[\ty{integer}, \ty{integer}] \to \ty{integer}$   & $-$ &  & \\
    \TT{multiplyInteger}          & $[\ty{integer}, \ty{integer}] \to \ty{integer}$   & $\times$ &  & \\
    \TT{divideInteger}            & $[\ty{integer}, \ty{integer}] \to \ty{integer}$   & $\divfn$   & Yes & \ref{note:integer-division-functions}\\
    \TT{modInteger}               & $[\ty{integer}, \ty{integer}] \to \ty{integer}$   & $\modfn$   & Yes & \ref{note:integer-division-functions}\\
    \TT{quotientInteger}          & $[\ty{integer}, \ty{integer}] \to \ty{integer}$   & $\quotfn$  & Yes & \ref{note:integer-division-functions}\\
    \TT{remainderInteger}         & $[\ty{integer}, \ty{integer}] \to \ty{integer}$   & $\remfn$   & Yes & \ref{note:integer-division-functions}\\
    \TT{equalsInteger}            & $[\ty{integer}, \ty{integer}] \to \ty{bool}$      & $=$ &  & \\
    \TT{lessThanInteger}          & $[\ty{integer}, \ty{integer}] \to \ty{bool}$      & $<$ &  & \\
    \TT{lessThanEqualsInteger}    & $[\ty{integer}, \ty{integer}] \to \ty{bool}$      & $\leq$ &  & \\
    %% Some of the signatures look like $ ... $ \text{\;\; $ ... $} to allow a break with some indentation afterwards
    \TT{appendByteString}         & $[\ty{bytestring}, \ty{bytestring}] $ \text{$\;\; \to \ty{bytestring}$}
                                           & $([c_1, \dots, c_m], [d_1, \ldots, d_n]) $ \text{$\;\; \mapsto [c_1,\ldots, c_m,d_1, \ldots, d_n]$} &  & \\
    \TT{consByteString} (Variant 1) & $[\ty{integer}, \ty{bytestring}] $ \text{$\;\; \to \ty{bytestring}$}
                                          & $(c,[c_1,\ldots,c_n]) $ \text{$\;\;\mapsto [\text{mod}(c,256) ,c_1,\ldots,c_{n}]$} &
                                          & \ref{note:consbytestring}\\
    \TT{consByteString} (Variant 2) & $[\ty{integer}, \ty{bytestring}] $ \text{$\;\; \to \ty{bytestring}$}
                                          & $(c,[c_1,\ldots,c_n])$ \text{$\;\;\mapsto
                                                       \begin{cases}
                                                          [c,c_1,\ldots,c_{n}] & \text{if $0 \leq c \leq 255$} \\
                                                         \errorX & \text{otherwise}
                                                       \end{cases}$} & Yes & \ref{note:consbytestring}\\
    \TT{sliceByteString}        & $[\ty{integer}, \ty{integer}, \ty{bytestring]} $  \text {$\;\; \to  \ty{bytestring}$}
                                                   &   $(s,k,[c_0,\ldots,c_n])$ \text{$\;\;\mapsto [c_{\max(s,0)},\ldots,c_{\min(s+k-1,n-1)}]$}
                                                   &  & \ref{note:slicebytestring}\\
    \TT{lengthOfByteString}       & $[\ty{bytestring}] \to \ty{integer}$ & $[] \mapsto 0, [c_1,\ldots, c_n] \mapsto n$ &  & \\
    \TT{indexByteString}          & $[\ty{bytestring}, \ty{integer}] $ \text{$\;\; \to \ty{integer}$}
                                                   & $([c_0,\ldots,c_{n-1}],j)$ \text{$\;\;\mapsto
                                                       \begin{cases}
                                                         c_i & \text{if $0 \leq j \leq n-1$} \\
                                                         \errorX & \text{otherwise}
                                                       \end{cases}$} & Yes & \\
    \TT{equalsByteString}         & $[\ty{bytestring}, \ty{bytestring}] $ \text{$\;\; \to \ty{bool}$}   & = &  & \ref{note:bytestring-comparison}\\
    \TT{lessThanByteString}       & $[\ty{bytestring}, \ty{bytestring}] $ \text{$\;\; \to \ty{bool}$}   & $<$ &  & \ref{note:bytestring-comparison}\\
    \TT{lessThanEqualsByteString} & $[\ty{bytestring}, \ty{bytestring}] $ \text{$\;\; \to \ty{bool}$}   & $\leq$ &  & \ref{note:bytestring-comparison}\\
    \TT{appendString}             & $[\ty{string}, \ty{string}] \to \ty{string}$
                                         & $([u_1, \dots, u_m], [v_1, \ldots, v_n]) $ \text{$\;\; \mapsto [u_1,\ldots, u_m,v_1, \ldots, v_n]$} &  & \\
    \TT{equalsString}             & $[\ty{string}, \ty{string}] \to \ty{bool}$           & = &  & \\
    \TT{encodeUtf8}               & $[\ty{string}] \to \ty{bytestring}$      & $\utfeight$ & & \ref{note:bytestring-encoding} \\
    \TT{decodeUtf8}               & $[\ty{bytestring}] \to \ty{string}$      & $\unutfeight$ & Yes & \ref{note:bytestring-encoding} \\
    \TT{sha2\_256}                & $[\ty{bytestring}] \to \ty{bytestring}$  & \text{Hash a $\ty{bytestring}$ using} \TT{SHA-}\TT{256}~\cite{FIPS-SHA2}. &  & \\
    \TT{sha3\_256}                & $[\ty{bytestring}] \to \ty{bytestring}$  & \text{Hash a $\ty{bytestring}$ using} \TT{SHA3-}\TT{256}~\cite{FIPS-SHA3}. &  & \\
    \TT{blake2b\_256}             & $[\ty{bytestring}] \to \ty{bytestring}$  & \text{Hash a $\ty{bytestring}$ using} \TT{Blake2b-}\TT{256}~\cite{IETF-Blake2}. &  & \\
    \TT{verifyEd25519Signature}          & $[\ty{bytestring}, \ty{bytestring}, $ \text{$\;\; \ty{bytestring}] \to \ty{bool}$}
    & Verify an \TT{Ed25519} digital signature. &  Yes
    & \ref{note:digital-signature-verification-functions}, \ref{note:ed25519-signature-verification}\\
    \TT{ifThenElse}               & $[\forall a_*, \ty{bool}, a_*, a_*] \to a_*$
                                                 & \text{$(\mathtt{true},t_1,t_2) \mapsto t_1$}
                                                 \text{$(\mathtt{false},t_1,t_2) \mapsto t_2$} & & \\
    \TT{chooseUnit}               & $[\forall a_*, \ty{unit}, a_*] \to a_*$        & $((), t) \mapsto t$ & & \\
    \TT{trace}                    & $[\forall a_*, \ty{string}, a_*] \to a_*$      & $ (s,t) \mapsto t$ &  & \ref{note:trace}\\
    \TT{fstPair}                  & $[\forall a_\#, \forall b_\#, \pairOf{a_\#}{b_\#}] \to a_\#$       & $(x,y) \mapsto x$ && \\
    \TT{sndPair}                  & $[\forall a_\#, \forall b_\#, \pairOf{a_\#}{b_\#}] \to b_\#$       & $(x,y) \mapsto y$ & & \\
    \TT{chooseList}               & $[\forall a_\#, \forall b_*, \listOf{a_\#}, b_*, b_*] \to b_*$
                                              & \text{$([], t_1, t_2) \mapsto t_1$,} \text{$([x_1,\ldots,x_n],t_1,t_2) \mapsto t_2\ (n \geq 1)$}. & & \\
    \TT{mkCons}                   & $[\forall a_\#, a_\#, \listOf{a_\#}] \to \listOf{a _\#}$  & $(x,[x_1,\ldots,x_n]) \mapsto [x,x_1,\ldots,x_n]$ &  & \\
    \TT{headList}                 & $[\forall a_\#, \listOf{a_\#}] \to a_\#$               & $[]\mapsto \errorX, [x_1,x_2, \ldots, x_n] \mapsto x_1$ & Yes & \\
    \TT{tailList}                 & $[\forall a_\#, \listOf{a_\#}] \to \listOf{a_\#}$
                                        &  \text{$[] \mapsto \errorX$,} \text{$ [x_1,x_2, \ldots, x_n] \mapsto [x_2, \ldots, x_n]$} & Yes & \\
    \TT{nullList}                 & $[\forall a_\#, \listOf{a_\#}] \to \ty{bool}$            & $ [] \mapsto \TT{true},
                                                                                                    [x_1,\ldots, x_n] \mapsto \TT{false}$& & \\
    \TT{chooseData}               & $[\forall a_*, \ty{data}, a_*, a_*, a_*, a_*, a_*] \to a_*$
    & $ (d,t_C, t_M, t_L, t_I, t_B) $
    \smallskip
    \newline  % The big \{ was abutting the text above
    \text{$\;\;\mapsto
               \left\{ \begin{array}{ll}  %% This looks better than `cases`
                 t_C  & \text{if $\is_C(d)$} \\
                 t_M  & \text{if $\is_M(d)$} \\
                 t_L  & \text{if $\is_L(d)$} \\
                 t_I  & \text{if $\is_I(d)$} \\
                 t_B  & \text{if $\is_B(d)$} \\
               \end{array}\right.$}  & & \\
    \TT{constrData}               & $[\ty{integer}, \listOf{\ty{data}}] \to \ty{data}$          & $\inj_C$ & & \\
    \TT{mapData}                  & $[\listOf{\pairOf{\ty{data}}{\ty{data}}}$ \text{$\;\; \to \ty{data}$}     & $\inj_M$& & \\
    \TT{listData}                 & $[\listOf{\ty{data}}] \to \ty{data} $      & $\inj_L$& & \\
    \TT{iData}                    & $[\ty{integer}] \to \ty{data} $            & $\inj_I$ & & \\
    \TT{bData}                    & $[\ty{bytestring}] \to \ty{data} $         & $\inj_B$& & \\
    \TT{unConstrData}             & $[\ty{data}]$ \text{$\;\; \to \pairOf{\ty{integer}}{\listOf{\ty{data}}}$} & $\proj_C$ & Yes& \\
    \TT{unMapData}                & $[\ty{data}]$ \text{$\;\; \to \listOf{\pairOf{\ty{data}}{\ty{data}}}$}  & $\proj_M$ & Yes& \\
    \TT{unListData}               & $[\ty{data}] \to \listOf{\ty{data}} $                          & $\proj_L$ & Yes& \\
    \TT{unIData}                  & $[\ty{data}] \to \ty{integer} $                                & $\proj_I$ & Yes& \\
    \TT{unBData}                  & $[\ty{data}] \to \ty{bytestring} $                             & $\proj_B$ & Yes& \\
    \TT{equalsData}               & $[\ty{data}, \ty{data}] \to \ty{bool} $                        & $ = $ & & \\
    \TT{mkPairData}               & $[\ty{data}, \ty{data}]$ \text{\;\; $\to \pairOf{\ty{data}}{\ty{data}}$}  & $(x,y) \mapsto (x,y) $ & & \\
    \TT{mkNilData}                & $[\ty{unit}] \to \listOf{\ty{data}} $                       & $() \mapsto []$ & & \\
    \TT{mkNilPairData}            & $[\ty{unit}] $ \text{$\;\; \to \listOf{\pairOf{\ty{data}}{\ty{data}}} $}   & $() \mapsto []$ & & \\
    \hline 
\end{longtable}

\kwxm{Maybe try \texttt{tabulararray} to see what sort of output that gives for the big table.}

\note{Integer division functions.}
\label{note:integer-division-functions}
We provide four integer division functions: \texttt{divideInteger},
\texttt{modInteger}, \texttt{quotientInteger}, and \texttt{remainderInteger},
whose denotations are mathematical functions $\divfn, \modfn, \quotfn$, and
$\remfn$ which are modelled on the corresponding Haskell operations. Each of
these takes two arguments and will fail (returning $\errorX$) if the second one
is zero.  For all $a,b \in \Z$ with $b \ne 0$ we have
$$
\divfn(a,b) \times b + \modfn(a,b) = a
$$
$$
  |\modfn(a,b)| < |b|
$$\noindent and
$$
  \quotfn(a,b) \times b + \remfn(a,b) = a
$$
$$
  |\remfn(a,b)| < |b|.
$$
\noindent The $\divfn$ and $\modfn$ functions form a pair, as do $\quotfn$ and $\remfn$;
$\divfn$ should not be used in combination with $\modfn$, not should $\quotfn$ be used
with $\modfn$.

For positive divisors $b$, $\divfn$ truncates downwards and $\modfn$ always
returns a non-negative result ($0 \leq \modfn(a,b) \leq b-1$).  The $\quotfn$
function truncates towards zero.  Table~\ref{table:integer-division-signs} shows
how the signs of the outputs of the division functions depend on the signs of
the inputs; $+$ means $\geq 0$ and $-$ means $\leq 0$, but recall that for $b=0$
all of these functions return the error value $\errorX$.
\begin{table}[H]
  \centering
    \begin{tabular}{|cc|cc|cc|}
        \hline
        a & b & $\divfn$ & $\modfn$ & $\quotfn$ & $\remfn$ \\
        \hline
        $+$ & $+$ & $+$ & $+$ & $+$ & $+$ \\
        $-$ & $+$ & $-$ & $+$ & $-$ & $-$ \\
        $+$ & $-$ & $-$ & $+$ & $+$ & $+$ \\
        $-$ & $-$ & $+$ & $-$ & $+$ & $-$ \\
        \hline
        \end{tabular}
   \caption{Behaviour of integer division functions}
   \label{table:integer-division-signs}
\end{table}
%% -------------------------------
%% |   n  d | div mod | quot rem |
%% |-----------------------------|
%% |  41  5 |  8   1  |   8   1  |
%% | -41  5 | -9   4  |  -8  -1  |
%% |  41 -5 | -9  -4  |  -8   1  |
%% | -41 -5 |  8  -1  |   8  -1  |
%% -------------------------------

\note{The \texttt{consByteString} function.}
\label{note:consbytestring}
In built-in semantics 1, the first argument of \texttt{consByteString} is an
arbitrary integer which will be reduced modulo 256 before being prepended to the
second argument.  In built-in semantics 2 we require that the first argument lies between 0
and 255 (inclusive): in any other case an error will occur.

\note{The \texttt{sliceByteString} function.}
\label{note:slicebytestring}
The application \texttt{[[(builtin sliceByteString) (con integer $s$)] (con
    integer $k$)] (con bytestring $b$)]} returns the substring of $b$ of length
$k$ starting at position $s$; indexing is zero-based, so a call with $s=0$
returns a substring starting with the first element of $b$, $s=1$ returns a
substring starting with the second, and so on.  This function always succeeds,
even if the arguments are out of range: if $b=[c_0, \ldots, c_{n-1}]$ then the
  application above returns the substring $[c_i, \ldots, c_j]$ where
  $i=\max(s,0)$ and $j=\min(s+k-1, n-1)$; if $j<i$ then the empty string is returned.

\note{Comparisons of bytestrings.}
\label{note:bytestring-comparison}
Bytestrings are ordered lexicographically in the usual way. If we have $a =
  [a_1, \ldots, a_m]$ and $b = [b_1, \ldots, b_n]$ then (recalling that if $m=0$
  then $a=[]$, and similarly for $b$),
\begin{itemize}
\item $a = b$ if and only if $m=n$ and $a_i = b_i$ for $1 \leq i \leq m$.

\item $a \leq b$ if and only if one of the following holds:
\begin{itemize}
  \item $a = []$
  \item $m,n > 0$ and $a_1 < b_1$
  \item $m,n > 0$ and $a_1 = b_1$ and $[a_2,\ldots,a_m] \leq [b_2,\ldots,b_n]$.
\end{itemize}
\item $a < b$ if and only if $a \leq b$ and $a \neq b$.
\end{itemize}
\noindent For example, $\mathtt{\#23456789} < \mathtt{\#24}$ and
$\mathtt{\#2345} < \mathtt{\#234500}$.  The empty bytestring is equal only to
itself and is strictly less than all other bytestrings.

\note{Encoding and decoding bytestrings.}
\label{note:bytestring-encoding}
The \texttt{encodeUtf8} and \texttt{decodeUtf8} functions convert between the
$\ty{string}$ type and the $\ty{bytestring}$ type.  We have defined
$\denote{\ty{string}}$ to consist of sequences of Unicode characters without
specifying any particular character representation, whereas
$\denote{\ty{bytestring}}$ consists of sequences of 8-bit bytes.  We define the
denotation of \texttt{encodeUtf8} to be the function
$$
\utfeight: \U^* \rightarrow \B^*
$$

\noindent which converts sequences of Unicode characters to sequences of bytes using the
well-known UTF-8 character encoding~\cite[Definition  D92]{Unicode-standard}.
The denotation of \texttt{decodeUtf8} is the partial inverse function

$$
\unutfeight: \B^* \rightarrow \U^*_{\errorX}.
$$

\noindent UTF-8 encodes Unicode characters encoded using between one and four
bytes: thus in general neither function will preserve the length of an object.
Moreover, not all sequences of bytes are valid representations of Unicode
characters, and \texttt{decodeUtf8} will fail if it receives an invalid input
(but \texttt{encodeUtf8} will always succeed).


\kwxm{In fact, strings are represented as sequences of UTF-16 characters, which
  use two or four bytes per character.  Do we need to mention that?  If we do,
  we'll need to be a little careful: there are sequences of 16-bit words that
  don't represent valid Unicode characters (for example, if the sequence uses
  surrogate codepoints improperly.  I don't think you can create a Haskell
  \texttt{Text} object (which is what our strings really are) that's invalid
  though.}


\note{Digital signature verification functions.}
\label{note:digital-signature-verification-functions}
We use a uniform interface for digital signature verification algorithms. A
digital signature verification function takes three bytestring arguments (in the
given order):
\begin{itemize}
  \item a public key $\vk$ (in this context $\vk$ is also known as a \textit{verification key}) 
  \item a message $m$
  \item a signature  $s$.
\end{itemize}
\noindent A signature verification function may require one
or more arguments to be well-formed in some sense (in particular an argument
may need to be of a specified length), and in this case the function will fail
(returning $\errorX$) if any argument is malformed. If all of the arguments are
well-formed then the verification function returns \texttt{true} if the private
key corresponding to $\vk$ was used to sign the message $m$ to produce $s$,
otherwise it returns \texttt{false}.

\note{Ed25519 signature verification.}
\label{note:ed25519-signature-verification}
The \texttt{verifyEd25519Signature}
function performs cryptographic
signature verification using the Ed25519 scheme \cite{ches-2011-24091,
  rfc8032-EdDSA}, and conforms to the interface described in
Note~\ref{note:digital-signature-verification-functions}.  The arguments must
have the following sizes:
\begin{itemize}
\item $\vk$: 32 bytes
\item $m$: unrestricted
\item $s$: 64 bytes.
\end{itemize}



\note{The \texttt{trace} function.}
\label{note:trace}
An application \texttt{[(builtin trace) $s$ $v$]} ($s$ a \texttt{string}, $v$
any Plutus Core value) returns $v$.  We do not specify the semantics any
further.  An implementation may choose to discard $s$ or to perform some
side-effect such as writing it to a terminal or log file.

\newpage

% I tried resetting the note number from V1-builtins here, but that made
% some hyperlinks wrong.  To get note numbers starting at one in each section, I
% think we have to define a new counter every time.
\newcounter{notenumberB}
\renewcommand{\note}[1]{
  \bigskip
  \refstepcounter{notenumberB}
  \noindent\textbf{Note \thenotenumberB. #1}
}

\subsection{Batch 2}
\label{sec:default-builtins-2}

\subsubsection{Built-in functions}
\label{sec:built-in-functions-2}
The second batch of builtin operations is defined in Table~\ref{table:built-in-functions-2}.

\setlength{\LTleft}{20mm}  % Shift the table right a bit to centre it on the page
\begin{longtable}[H]{|l|l|l|c|c|}
    \hline
    \text{Function} & \text{Signature} & \text{Denotation} & \text{Can} & \text{Note} \\
    & & & fail? & \\
    \hline
    \endfirsthead
    \hline
    \text{Function} & \text{Type} & \text{Denotation} & \text{Can} & \text{Note}\\
    & & & fail? & \\
    \hline
    \endhead
    \hline
    \caption{Built-in Functions}
    \endfoot
    \caption[]{Built-in Functions}
    \label{table:built-in-functions-2}
    \endlastfoot
    \TT{serialiseData}                        & $[\ty{data}] \to \ty{bytestring}$   &  $\mathcal{E}_{\mathtt{data}}$ &
      & \ref{note:serialise-data}\\
\hline
\end{longtable}

\note{Serialising $\ty{data}$ objects.}
\label{note:serialise-data}
The \texttt{serialiseData} function takes a $\ty{data}$ object and converts it
into a bytestring using a CBOR encoding.  A full specification of the encoding
(including the definition of $\mathcal{E}_{\mathtt{data}}$) is provided in
Appendix~\ref{appendix:data-cbor-encoding}.

% I tried resetting the note number from V1-builtins here, but that made
% some hyperlinks wrong.  To get note numbers starting at one in each section, I
% think we have to define a new counter every time.
\newcounter{notenumberC}
\renewcommand{\note}[1]{
  \bigskip
  \refstepcounter{notenumberC}
  \noindent\textbf{Note \thenotenumberC. #1}
}

\subsection{Batch 3}
\label{sec:default-builtins-3}

\subsubsection{Built-in functions}
\label{sec:built-in-functions-3}
The third batch of builtin operations is defined in Table~\ref{table:built-in-functions-3}.

\setlength{\LTleft}{-10mm}  % Shift the table left a bit to centre it on the page
\begin{longtable}[H]{|l|p{42mm}|p{35mm}|c|c|}
    \hline
    \text{Function} & \text{Signature} & \text{Denotation} & \text{Can} & \text{Note} \\
    & & & fail? & \\
    \hline
    \endfirsthead
    \hline
    \text{Function} & \text{Type} & \text{Denotation} & \text{Can} & \text{Note}\\
    & & & fail? & \\
    \hline
    \endhead
    \hline
    \caption{Built-in Functions}
    \endfoot
    \caption[]{Built-in Functions}
    \label{table:built-in-functions-3}
    \endlastfoot
    \TT{verifyEcdsaSecp256k1Signature}        & $[\ty{bytestring}, \ty{bytestring}, $ \text{$\;\; \ty{bytestring}] \to \ty{bool}$}
        & Verify an SECP-256k1 ECDSA signature & Yes & \ref{note:verify-ecdsa-secp256k1-signature}\\
    \TT{verifySchnorrSecp256k1Signature}      & $[\ty{bytestring}, \ty{bytestring}, $ \text{$\;\; \ty{bytestring}] \to \ty{bool}$}
          & Verify an SECP-256k1 Schnorr signature & Yes & \ref{note:verify-schnorr-secp256k1-signature}\\
\hline
\end{longtable}


\note{Secp256k1 ECDSA Signature verification.}
\label{note:verify-ecdsa-secp256k1-signature}
The \texttt{verifyEcdsaSecp256k1Signature} function performs elliptic curve
digital signature verification \cite{ANSI-X9.62, ANSI-x9.142,
  Johnson-Menezes-Vanstone-ECDSA} over the \texttt{secp256k1}
curve~\cite[\S2.4.1]{SECP256} and conforms to the interface described in
Note~\ref{note:digital-signature-verification-functions} of
Section~\ref{sec:built-in-functions-1}.  The arguments must have the
following sizes:
\begin{itemize}
\item $\vk$: 33 bytes
\item $m$: 32 bytes
\item $s$: 64 bytes.
\end{itemize}
The public key $\vk$ is expected to be in the 33-byte compressed form described
in~\cite{Bitcoin-ECDSA}.  Moreover, the ECDSA scheme admits two distinct valid
signatures for a given message and private key, and  we follow the restriction
imposed by Bitcoin (see~\cite{BIP-146},
\texttt{LOW\_S}) and \textbf{only accept the smaller signature};
\texttt{verifyEcdsa\-Secp\-256k1Signature} will return $\false$ if the larger
one is supplied.

% For more on the lower signature business, see
%    https://github.com/IntersectMBO/plutus/pull/4591#issuecomment-1120797931
% and
%    https://github.com/IntersectMBO/plutus/pull/4591#issuecomment-1121684776
% The C code that performs the check in the bitcoin implementation is at
%   https://github.com/bitcoin-core/secp256k1/blob/485f608fa9e28f132f127df97136617645effe81/src/secp256k1.c#L400
% and
%   https://github.com/bitcoin-core/secp256k1/blob/485f608fa9e28f132f127df97136617645effe81/src/scalar_low_impl.h#L85
%
% Schnorr signatures do something substantially different for ECDSA and I don't think
% the question of multiple signatures arises for verifySchnorrSecp256k1Signature.

\note{Secp256k1 Schnorr Signature verification.}
\label{note:verify-schnorr-secp256k1-signature}
The \texttt{verifySchnorrSecp256k1Signature} function performs verification of
Schnorr signatures~\cite{Schnorr89, BIP-340} over the \texttt{secp256k1} curve
and conforms to the interface described in
Note~\ref{note:digital-signature-verification-functions} of
Section~\ref{sec:built-in-functions-1}.  The arguments are expected to be
of the forms specified in BIP-340~\cite{BIP-340} and thus should have the
following sizes:
\begin{itemize}
\item $\vk$: 32 bytes
\item $m$: unrestricted
\item $s$: 64 bytes.
\end{itemize}

% I tried resetting the note number from V1-builtins here, but that made
% some hyperlinks wrong.  To get note numbers starting at one in each section, I
% think we have to define a new counter every time.
\newcounter{notenumberD}
\renewcommand{\note}[1]{
  \bigskip
  \refstepcounter{notenumberD}
  \noindent\textbf{Note \thenotenumberD. #1}
}

\newcommand{\itobsBE}{\mathsf{itobs_{BE}}}
\newcommand{\itobsLE}{\mathsf{itobs_{LE}}}
\newcommand{\bstoiBE}{\mathsf{bstoi_{BE}}}
\newcommand{\bstoiLE}{\mathsf{bstoi_{LE}}}

\subsection{Batch 4}
\label{sec:default-builtins-4}
The fourth batch of built-in types and functions adds support for
\begin{itemize}
\item The \texttt{Blake2b-224} and \texttt{Keccak-256} hash functions (see~\cite{CIP-0101}).
\item Conversion functions from integers to bytestrings and vice-versa (see~\cite{CIP-0121}).
\item BLS12-381 elliptic curve pairing operations
(see~\cite{CIP-0381}, \cite{BLS12-381}, \cite[4.2.1]{IETF-pairing-friendly-curves}, \cite{BLST-library}).
 For clarity these are described separately in Sections~\ref{sec:bls-types-4} and \ref{sec:bls-builtins-4}.
\end{itemize}

\subsubsection{Miscellaneous built-in functions}
\label{sec:misc-builtins-4}

\setlength{\LTleft}{-17mm} % Shift the table left a bit to centre it on the page
\begin{longtable}[H]{|l|p{45mm}|p{65mm}|c|c|}
    \hline \text{Function} & \text{Signature} & \text{Denotation} & \text{Can}
    & \text{Note} \\ & & & fail?
    & \\ \hline \endfirsthead \hline \text{Function} & \text{Type}
    % This caption goes on every page of the table except the last.  Ideally it
    % would appear only on the first page and all the rest would say
    % (continued). Unfortunately it doesn't seem to be easy to do that in a
    % longtable.
    \endfoot
%%    \caption[]{Built-in Functions}
    \caption[]{Miscellaneous built-in Functions}
    \label{table:misc-built-in-functions-4}
    \endlastfoot
%% G1
    \hline
    \TT{blake2b\_224} & $[\ty{bytestring}] \to \ty{bytestring}$  & \text{Hash a $\ty{bytestring}$ using
                                                                   \TT{Blake2b-224}~\cite{IETF-Blake2}} & No & \\[2mm]
    \TT{keccak\_256}  & $[\ty{bytestring}] \to \ty{bytestring}$  & \text{Hash a $\ty{bytestring}$ using
                                                                   \TT{Keccak-256}~\cite{KeccakRef3}} & No & \\
    \hline\strut
    \TT{integerToByteString} & $[\ty{bool}, \ty{integer}, \ty{integer}]$  \text{\: $\to \ty{bytestring}$}
                                        & $(e, w, n) $ \text{$\mapsto \begin{cases}
                                        \itobsLE(w,n) & \text{if $e=\false$}\\
                                        \itobsBE(w,n) & \text{if $e=\true$}\\
                                        \end{cases}$}
                                        & Yes & \ref{note:itobs}\strut \\[6mm]
    \TT{byteStringToInteger} & $[\ty{bool}, \ty{bytestring}] $ \text{\: $ \to \ty{bytestring}$}
                                        & $(e, [c_0, \ldots, c_{N-1}]) $ \text{\; $\mapsto \begin{cases}
                                        \sum_{i=0}^{N-1}c_{i}256^i & \text{if $e=\false$}\\
                                        \sum_{i=0}^{N-1}c_{i}256^{N-1-i} & \text{if $e=\true$}\\
                                        \end{cases}$}
                                        &  No & \ref{note:bstoi}\\[12mm]
    \hline
\end{longtable}

\note{Integer to bytestring conversion.}
\label{note:itobs}
The \texttt{integerToByteString} function converts non-negative integers to bytestrings.
It takes three arguments:
\begin{itemize}
\item A boolean endianness flag $e$.
\item An integer width argument $w$ with $0 \leq w < 8192$.
\item The integer $n$ to be converted: it is required that $0 \leq n < 256^{8192} = 2^{65536}$.
\end{itemize}

\noindent The conversion is little-endian ($\mathsf{LE}$) if $e$ is
\texttt{(con bool False)} and big-endian ($\mathsf{BE}$) if $e$ is
\texttt{(con bool True)}. If the width $w$
is zero then the output is a bytestring which is just large enough to hold the
converted integer.  If $w>0$ then the output is exactly $w$ bytes long, and it
is an error if $n$ does not fit into a bytestring of that size; if necessary,
the output is padded with \texttt{0x00} bytes (on the right in the little-endian
case and the left in the big-endian case) to make it the correct length.  For
example, the five-byte little-endian representation of the
integer \texttt{0x123456} is the bytestring \texttt{[0x56, 0x34, 0x12, 0x00,
0x00]} and the five-byte big-endian representation is \texttt{[0x00, 0x00, 0x12,
0x34, 0x56]}.  In all cases an error occurs error if $w$ or $n$ lies outside the
expected range, and in particular if $n$ is negative.

\newpage
\noindent The precise semantics of \texttt{integerToByteString} are given
by the functions $\itobsLE: \Z \times \Z \rightarrow \withError{\B^*}$ and $\itobsBE
: \Z \times \Z \rightarrow \withError{\B^*}$.  Firstly we deal with out-of-range cases and
the case $n=0$:

$$
\itobsLE (w,n) = \itobsBE (w,n) = 
\begin{cases}
  \errorX & \text{if $n<0$ or $n \geq 2^{65536}$}\\
  \errorX & \text{if $w<0$ or $w > 8192$}\\
  [] & \text{if $n=0$ and $0 \leq w \leq 8192$}\\
\end{cases}
$$

\noindent  Now assume that none of the conditions above hold, so $0 < n < 2^{65536}$ and
$0 \leq w \leq 8192$.  Since $n>0$ it has a unique base-256 expansion of the
form $n = \sum_{i=0}^{N-1}a_{i}256^i$ with $N \geq 1$, $a_i \in \B$ for $0 \leq
i \leq N-1$ and $a_{N-1} \ne 0$.  We then have

$$
\itobsLE (w,n) =
\begin{cases}
  [a_0, \ldots, a_{N-1}] & \text{if $w=0$} \\
  [b_0, \ldots, b_{w-1}] &  \text{if $w > 0$ and $N\leq w$, where }
      b_i = \begin{cases}
                a_i & \text{if $i \leq N-1$} \\
                0   & \text{if $i \geq N$}\\
            \end{cases}\\
  \errorX & \text{if $w > 0$ and $N > w$}
\end{cases}
$$

\noindent and

$$
\noindent
\itobsBE (w,n) =
\begin{cases}
  [a_{N-1}, \ldots, a_0] & \text{if $w=0$} \\
  [b_0, \ldots, b_{w-1}] &  \text{if $w > 0$ and $N\leq w$, where }
      b_i = \begin{cases}
                0   & \text{if $i \leq w-1-N$}\\
                a_{w-1-i} & \text{if $i \geq w-N$} \\
            \end{cases}\\
  \errorX & \text{if $w > 0$ and $N > w$.}
\end{cases}
$$

\note{Bytestring to integer conversion.}
\label{note:bstoi}
The \texttt{byteStringToInteger} function converts bytestrings to non-negative
integers.  It takes two arguments:
\begin{itemize}
\item A boolean endianness flag $e$.
\item The bytestring $s$ to be converted.
\end{itemize}
\noindent  
The conversion is little-endian if $e$ is \texttt{(con bool False)} and
big-endian if $e$ is \texttt{(con bool True)}. In both cases the empty bytestring is
converted to the integer 0. All bytestrings are legal inputs and there is no
limitation on the size of $s$.

\subsubsection{BLS12-381 built-in types}
\label{sec:bls-types-4}

\noindent Supporting the BLS12-381 operations involves adding three new types
and seventeen new built-in functions.  The description of the semantics of these
types and functions is quite complex and requires a considerable amount of
notation, most of which is used only in Sections~\ref{sec:bls-types-4} and~\ref{sec:bls-builtins-4}.

\bigskip
\noindent Table~\ref{table:built-in-types-4} describes three new built-in
types.

\newcommand{\TyMlResult}{\ty{bls12\_381\_mlresult}}
\newcommand{\MlResultDenotation}{H}
\newcommand{\Fq}{\mathbb{F}_q}
\newcommand{\Fqq}{\mathbb{F}_{q^2}}
\newcommand{\FF}{\mathbb{F}_{q^{12}}}

\begin{table}[H]
  \centering
    \begin{tabular}{|l|p{2cm}|l|}
        \hline
        Type & Denotation & Concrete Syntax\\
        \hline
        $\ty{bls12\_381\_G1\_element}$ &   $G_1$ & \texttt{0x[0-9A-Fa-f]\{96\}} \text{(see Note~\ref{note:bls-syntax})}\\
        $\ty{bls12\_381\_G2\_element}$ &   $G_2$ & \texttt{0x[0-9A-Fa-f]\{192\}} \text{(see Note~\ref{note:bls-syntax})}\\
        $\TyMlResult$    &   $\MlResultDenotation$  &  None (see Note~\ref{note:bls-syntax})\\
        \hline
    \end{tabular}
    \caption{Atomic Types}
    \label{table:built-in-types-4}
\end{table}

%% \paragraph{$G_1$ and $G_2$}.
\noindent Here $G_1$ and  $G_2$ are both additive cyclic groups of prime order $r$, where 
$$
r = \mathtt{0x73eda753299d7d483339d80809a1d80553bda402fffe5bfeffffffff00000001}.
$$
        
\paragraph{The fields $\Fq$ and $\Fqq$.}
\noindent To define the groups $G_1$ and $G_2$ we need the finite field $\Fq$ where
\begin{align*}
q = \mathtt{0x}&\mathtt{1a0111ea397fe69a4b1ba7b6434bacd764774b84f38512bf}\\
              &\mathtt{6730d2a0f6b0f6241eabfffeb153ffffb9feffffffffaaab}
\end{align*}

\noindent which is a 381-bit prime. The field $\Fq$ is isomorphic to $\Z_q$,
the ring of integers modulo $q$, and hence there is a natural epimorphism from
$\Z$ to $\Fq$ which we denote by $n \mapsto \bar{n}$.  Given $x \in \Fq$, we
denote by $\tilde{x}$ the smallest non-negative integer $n$ with $\bar{n} = x$.
We sometimes use literal integers to represent elements of $\Fq$ in the obvious
way.

We also make use of the field $\Fqq = \Fq[X]/(X^2+1)$; we may regard $\Fqq$ as
the set $\{a+bu: a,b \in \Fq\}$ where $u^2=-1$.

It is convenient to say that an element $a$ of $\Fq$ is \textit{larger} than
another element $b$ (and $b$ is \textit{smaller} than $a$) if $\tilde{a}
> \tilde{b}$ in $\Z$.  We extend this terminology to $\Fqq$ by saying that
$a+bu$ is larger than $c+du$ if either $b$ is larger than $d$ or $b=d$ and $a$
is larger than $c$.


\paragraph{The groups $G_1$ and $G_2$.}
\noindent There are elliptic curves $E_1$ defined over $\Fq$:
$$
E_1: Y^2 = X^3 + 4
$$

\noindent and $E_2$ defined over $\Fqq$:
$$
E_2: Y^2 = X^3 + 4(u+1).
$$

\noindent $E_1(\Fq)$ and  $E_2(\Fqq)$  are abelian groups under the
usual elliptic curve addition operations as described
in~\cite[III.2]{Silverman-Arithmetic-EC} or~\cite[2.1]{Costello-pairings}.
$G_1$ is a subgroup of $E_1(\Fq)$ and $G_2$ is a subgroup of $E_2(\Fqq)$;
explicit generators for $G_1$ and $G_2$ are given
in~\cite[4.2.1]{IETF-pairing-friendly-curves}.  We denote the identity element
(the point at infinity) in $G_1$ by $\mathcal{O}_{G_1}$ and that in $G_2$ by
$\mathcal{O}_{G_2}$.  Given an integer $n$ and a group element $a$ in $G_1$ or
$G_2$, the scalar multiple $na$ is defined as usual to be $a + \cdots + a$ ($n$
times) if $n>0$ and $-a + \cdots + -a$ ($-n$ times) if $n<0$; $0a$ is the
identity element of the group.

\paragraph{The \texttt{bls12\_381\_MlResult} type.}
\noindent Values of the \texttt{bls12\_381\_MlResult} type are completely
opaque and can only be obtained as a result of \texttt{bls12\_381\_millerLoop}
or by multiplying two existing elements of type \texttt{bls12\_381\_MlResult}.
We provide neither a serialisation format nor a concrete syntax for values of
this type: they exist only ephemerally during computation.  We do not specify
$\MlResultDenotation$, the denotation of $\TyMlResult$, precisely, but it
must be a multiplicative abelian group. See Note~\ref{note:pairing} for more on
this.

\newpage
\subsubsection{BLS12-381 built-in functions}
\label{sec:bls-builtins-4}

\newcommand{\hash}{\mathsf{hash}}
\newcommand{\compress}{\mathsf{compress}}
\newcommand{\uncompress}{\mathsf{uncompress}}

\setlength{\LTleft}{-4mm} % Shift the table left a bit to centre it on the page
\begin{longtable}[H]{|l|p{5cm}|p{25mm}|c|c|}
    \hline \text{Function} & \text{Signature} & \text{Denotation} & \text{Can}
    & \text{Note} \\ & & & fail?
    & \\ \hline \endfirsthead \hline \text{Function} & \text{Type}
    & \text{Denotation} & \text{Can} & \text{Note}\\ & & & fail?
    & \\ \hline \endhead \hline \caption{BLS12-381 built-in Functions}
    % This caption goes on every page of the table except the last.  Ideally it
    % would appear only on the first page and all the rest would say
    % (continued). Unfortunately it doesn't seem to be easy to do that in a
    % longtable.
    \endfoot
%%    \caption[]{Built-in Functions}
    \caption[]{BLS12-381 built-in Functions (continued)}
    \label{table:built-in-functions-4}
    \endlastfoot
%% G1
    \TT{bls12\_381\_G1\_add}  &
    $[ \ty{bls12\_381\_G1\_element}$,
      \text{\; $\ty{bls12\_381\_G1\_element} ]$}
      \text{\: $ \to \ty{bls12\_381\_G1\_element}$} & $(a,b) \mapsto a+b$  &  No & \\[2mm]
    \TT{bls12\_381\_G1\_neg}  &
      $ [ \ty{bls12\_381\_G1\_element} ]$  \text{\;\; $\to \ty{bls12\_381\_G1\_element}$} & $a \mapsto -a$  & No & \\[2mm]
    \TT{bls12\_381\_G1\_scalarMul}  &
    $[ \ty{integer}$,
      \text{\; $\ty{bls12\_381\_G1\_element} ]$}
      \text{\: $ \to \ty{bls12\_381\_G1\_element}$} & $(n,a) \mapsto na$ &  No & \\[2mm]
    \TT{bls12\_381\_G1\_equal}  &
    $[ \ty{bls12\_381\_G1\_element}$,
      \text{\; $\ty{bls12\_381\_G1\_element} ]$}
      \text{\: $ \to \ty{bool}$} & $=$ &  No & \\[2mm]
    \TT{bls12\_381\_G1\_hashToGroup}  &
    $[ \ty{bytestring}, \ty{bytestring}]$
      \text{\: $ \to \ty{bls12\_381\_G1\_element}$} & $\hash_{G_1}$ &  Yes & \ref{note:hashing-into-group}\\[2mm]
    \TT{bls12\_381\_G1\_compress}  &
    $[\ty{bls12\_381\_G1\_element}]$
      \text{\: $ \to \ty{bytestring}$} & $\compress_{G_1}$  &  No & \ref{note:group-compression}\\[2mm]
    \TT{bls12\_381\_G1\_uncompress}  &
    $[ \ty{bytestring}]$
      \text{\: $ \to \ty{bls12\_381\_G1\_element}$} & $\uncompress_{G_1}$  &  Yes & \ref{note:group-uncompression}\\[2mm]
    \hline 
%% G2
    \TT{bls12\_381\_G2\_add}  &
    $[ \ty{bls12\_381\_G2\_element}$,
      \text{\; $\ty{bls12\_381\_G2\_element} ]$}
      \text{\: $ \to \ty{bls12\_381\_G2\_element}$} & $(a,b) \mapsto a+b$ &  No & \\[2mm]
    \TT{bls12\_381\_G2\_neg}  &
      $ [ \ty{bls12\_381\_G2\_element} ]$  \text{\;\; $\to \ty{bls12\_381\_G2\_element}$} & $a \mapsto -a$  & No & \\[2mm]
    \TT{bls12\_381\_G2\_scalarMul}  &
    $[ \ty{integer}$,
      \text{\; $\ty{bls12\_381\_G2\_element} ]$}
      \text{\: $ \to \ty{bls12\_381\_G2\_element}$} & $(n,a) \mapsto na$ &  No & \\[2mm]
    \TT{bls12\_381\_G2\_equal}  &
    $[ \ty{bls12\_381\_G2\_element}$,
      \text{\; $\ty{bls12\_381\_G2\_element} ]$}
      \text{\: $ \to \ty{bool}$} & $=$ &  No & \\[2mm]
    \TT{bls12\_381\_G2\_hashToGroup}  &
    $[ \ty{bytestring}, \ty{bytestring}]$
      \text{\: $ \to \ty{bls12\_381\_G2\_element}$} & $\hash_{G_2}$  &  Yes & \ref{note:hashing-into-group}\\[2mm]
    \TT{bls12\_381\_G2\_compress}  &
    $[\ty{bls12\_381\_G2\_element}]$
      \text{\: $ \to \ty{bytestring}$} & $\compress_{G_2}$  &  No & \ref{note:group-compression}\\[2mm]
    \TT{bls12\_381\_G2\_uncompress}  &
    $[ \ty{bytestring}]$
      \text{\: $ \to \ty{bls12\_381\_G2\_element}$} & $\uncompress_{G_2}$  &  Yes & \ref{note:group-uncompression}\\[2mm]
    \hline 
    \TT{bls12\_381\_millerLoop}  &
    $[ \ty{bls12\_381\_G1\_element}$,
      \text{\; $\ty{bls12\_381\_G2\_element} ]$}
    \text{\: $ \to \TyMlResult$} & $e$ &  No & \ref{note:pairing}\\[2mm]
    \TT{bls12\_381\_mulMlResult}  &
    $[ \TyMlResult$,
    \text{\; $\TyMlResult]$}
    \text{\: $\to \TyMlResult$} & $(a,b) \mapsto ab$ & No & \ref{note:pairing}\\[2mm]
    \TT{bls12\_381\_finalVerify}  &
    $[ \TyMlResult$,
    \text{\; $\TyMlResult] \to \ty{bool}$} & $\phi$ & No & \ref{note:pairing}\\[2mm]
    \hline
\end{longtable}


\note{Hashing into $G_1$ and $G_2$.}
\label{note:hashing-into-group}
The denotations $\hash_{G_1}$ and $\hash_{G_2}$
of \texttt{bls12\_381\_G1\_hashToGroup} and
\texttt{bls12\_381\_G2\_hashToGroup} both take an arbitrary bytestring $b$ (the
\textit{message}) and a (possibly empty) bytestring of length at most 255 known as a \textit{domain
separation tag} (DST)~\cite[2.2.5]{IETF-hash-to-curve} and hash them to obtain a
point in $G_1$ or $G_2$ respectively.  The details of the hashing process are
described in~\cite{IETF-hash-to-curve} (see specifically Section 8.8), except
that
\textbf{we do not support DSTs of length greater than 255}: an attempt to use a
longer DST directly will cause an error.  If a longer DST is required then it
should be hashed to obtain a short DST as described
in~\cite[5.3.3]{IETF-hash-to-curve}, and then this should be supplied as the
second argument to the appropriate \texttt{hashToGroup} function.

%% Some hashing
%% implementations also allow a third argument (an ``augmentation string''), but we
%% do not support this since the same effect can be obtained by appending
%% (prepending?) the augmentation string to the message before hashing.

\newcommand{\ymin}{y_{\text{min}}}
\newcommand{\ymax}{y_{\text{max}}}

\note{Compression for elements of $G_1$ and $G_2$.} 
\label{note:group-compression}
Points in $G_1$ and $G_2$ are encoded as bytestrings in a ``compressed'' format
where only the $x$-coordinate of a point is encoded and some metadata is used to
indicate which of two possible $y$-coordinates the point has.  The encoding
format is based on the Zcash encoding for BLS12-381 points:
see~\cite{Zcash-serialisation} or~\cite[``Serialization'']{BLST-library}
or~\cite[Appendices C and D]{IETF-pairing-friendly-curves}.  In detail,

\begin{itemize}

\item Given an element $x$ of $\Fq$, $\tilde{x}$ can be written as a 381-bit
binary number: $\tilde{x} = \sum_{i=0}^{380}b_i2^i$ with $b_i \in \{0,1\}$.  We
define $\mathsf{bits}(x)$ to be the bitstring $b_{380}\cdots b_0$.

\item A non-identity element of $G_1$ can be written in the form $(x,y)$ with $x,y\in\Fq$.
Not every element $x$ of $\Fq$ is the $x$-coordinate of a point in $G_1$, but
those which are in fact occur as the $x$-coordinate of two distinct points in
$G_1$ whose $y$-coordinates are the negatives of each other.  A similar
statement is true for $\Fqq$ and $G_2$.  In both cases we denote the smaller of
the possible $y$-coordinates by $\ymin(x)$ and the larger by $\ymax(x)$.

\item For $(x,y) \in G_1\backslash \mathcal{O}_{G_1}$ we define
$$
\compress_{G_1} (x,y) = \begin{cases}
\mathsf{1}\mathsf{0}\mathsf{0}\cdot\mathsf{bits}(x) & \text{if $y=\ymin(x)$}\\
\mathsf{1}\mathsf{0}\mathsf{1}\cdot\mathsf{bits}(x) & \text{if $y=\ymax(x)$}\\
\end{cases}
$$
\item We encode the identity element of $G_1$ using
$$
\compress_{G_1}(\mathcal{O}_{G_1}) = \mathsf{1}\mathsf{1}\mathsf{0}\cdot\mathsf{0}^{381},
$$
\noindent where $\mathsf{0}^{381}$ denotes a string of 381 $\mathsf{0}$ bits.
\end{itemize}
\noindent Thus in all cases the encoding of an element of $G_1$ requires exactly 384 bits,
or 48 bytes.

\medskip 

\noindent 
\begin{itemize}
\item Similarly, every non-identity element of $G_2$ can be written
in the form $(x,y)$ with $x,y \in \Fqq$.  We define

$$
\compress_{G_2} (a+bu,y) = \begin{cases}
\mathsf{1}\mathsf{0}\mathsf{0}\cdot\mathsf{bits}(b)\cdot\mathsf{0}\mathsf{0}\mathsf{0}\cdot\mathsf{bits}(a)
& \text{if $y=\ymin(a+bu)$}\\
\mathsf{1}\mathsf{0}\mathsf{1}\cdot\mathsf{bits}(b)\cdot\mathsf{0}\mathsf{0}\mathsf{0}\cdot\mathsf{bits}(a) &
 \text{if $y=\ymax(a+bu)$}\\
\end{cases}
$$

\item The identity element of $G_2$ is encoded using
$$
\compress_{G_2}(\mathcal{O}_{G_2}) = \mathsf{1}\mathsf{1}\mathsf{0}\cdot\mathsf{0}^{765}.
$$

\end{itemize}
\noindent The encoding of an element of $G_2$ requires exactly 768 bits, or 96 bytes.

Note that in all cases the most significant bit of a compressed point is 1.  In
the Zcash serialisation scheme this indicates that the point is compressed;
Zcash also supports a serialisation format where both the $x$- and
$y$-coordinates of a point are encoded, and in that case the leading bit of the
encoded point is 0.  We do not support this format.

\note{Uncompression for elements of $G_1$ and $G_2$.} 
\label{note:group-uncompression}
There are two (partial) ``uncompression'' functions $\uncompress_{G_1}$ and
$\uncompress_{G_2}$ which convert bytestrings into group elements; these are
obtained by inverting the process described in
Note~\ref{note:group-compression}.

\paragraph{Uncompression for $G_1$ elements.}  Given a bytestring $b$, it is checked that
$b$ contains exactly 48 bytes.  If not, then $\uncompress_{G_1}(b) = \errorX$ (ie,
uncompression fails).  If the length is equal to 48 bytes, write $b$ as a
sequence of bits: $b = b_{383} \cdots b_0$.
\begin{itemize}
\item If $b_{383} \neq 1$, then $\uncompress_{G_1}(b) = \errorX$.
\item If $b_{383} = b_{382} = 1$ then
$\uncompress_{G_1}(b) =
\begin{cases}
\mathcal{O}_{G_1} & \text{if $b_{381} = b_{380} = \cdots = b_0 = 0$}\\
\errorX & \text{otherwise}.
\end{cases}$
\item If $b_{383}=1$ and $b_{382}=0$, let $c=\sum_{i=0}^{380}b_i2^i \in \N$.
\begin{itemize}
\item If $c \geq q$, $\uncompress_{G_1}(b) = \errorX$.
\item Otherwise, let $x = \bar{c} \in \Fq$ and let $z = x^3+4$. If $z$ is not a
square in $\Fq$, then $\uncompress_{G_1}(b) = \errorX$.
\item If $z$ is a square then let
$y=\begin{cases}
\ymin(x) & \text{if $b_{381}=0$}\\
\ymax(x) & \text{if $b_{381}=1$}.
\end{cases}$
\item Then $\uncompress_{G_1}(b) = \begin{cases}
(x,y) & \text{if $(x,y) \in G_1$}\\
\errorX & \text{otherwise}.
\end{cases}$
\end{itemize}
\end{itemize}


\paragraph{Uncompression for $G_2$ elements.}  Given a bytestring $b$, it is checked that
$b$ contains exactly 96 bytes.  If not, then $\uncompress_{G_2}(b) = \errorX$ (ie,
uncompression fails).  If the length is equal to 96 bytes, write $b$ as a
sequence of bits: $b = b_{767} \cdots b_0$.
\begin{itemize}
\item If $b_{767} \neq 1$, then $\uncompress_{G_2}(b) = \errorX$.
\item If $b_{767} = b_{766} = 1$ then $\uncompress_{G_2}(b) =
\begin{cases}
\mathcal{O}_{G_2} & \text{if $b_{765} = b_{764} = \cdots = b_0 = 0$}\\
\errorX & \text{otherwise}.
\end{cases}$
\item If $b_{767}=1$ and $b_{766} = 0$, let $c=\sum_{i=0}^{383}b_i2^i$ and $d=\sum_{i=384}^{764}b_i2^{i-384} \in \N$.
\begin{itemize}
\item If $c \geq q$ or $d \geq q$, $\uncompress_{G_2}(b) = \errorX$.
\item Otherwise, let $x = \bar{c}+\bar{d}u \in \Fqq$ and let $z = x^3+4(u+1)$.
If $z$ is not a square in $\Fqq$, then $\uncompress_{G_2}(b) = \errorX$.
\item If $z$ is a square then let
$y=\begin{cases}
\ymin(x) & \text{if $b_{765}=0$}\\
\ymax(x) & \text{if $b_{765}=1$}.
\end{cases}$
\item Then $\uncompress_{G_2}(b) = \begin{cases}
(x,y) & \text{if $(x,y) \in G_2$}\\
\errorX & \text{otherwise}.
\end{cases}$
\end{itemize}
\end{itemize}


\note{Concrete syntax for BLS12-381 types.}
\label{note:bls-syntax}
Concrete syntax for the $\ty{bls12\_381\_G1\_element}$ and
$\ty{bls12\_381\_G2\_element}$ types is provided via the compression and
decompression functions defined in Notes~\ref{note:group-compression}
and~\ref{note:group-uncompression}.  Specifically, a value of type
$\ty{bls12\_381\_G1\_element}$ is denoted by a term of the form \texttt{(con
bls12\_381\_G1\_element 0x...)} where \texttt{...}  consists of 96 hexadecimal
digits representing the 48-byte compressed form of the relevant point.
Similarly, a value of type $\ty{bls12\_381\_G2\_element}$ is denoted by a term
of the form \texttt{(con bls12\_381\_G2\_element 0x...)}  where \texttt{...}
consists of 192 hexadecimal digits representing the 96-byte compressed form of
the relevant point.  \textbf{This syntax is provided only for use in textual
Plutus Core programs}, for example for experimentation and testing.  We do not
support constants of any of the BLS12-381 types in serialised programs on the
Cardano blockchain: see Section~\ref{sec:flat-constants}.  However, for
$\ty{bls12\_381\_G1\_element}$ and $\ty{bls12\_381\_G2\_element}$ one can use
the appropriate uncompression function on a  bytestring constant at runtime:
for example, instead of
$$
\texttt{(con bls12\_381\_G1\_element 0xa1e9a0...)}
$$
write
$$
\texttt{[(builtin bls12\_381\_G1\_uncompress) (con bytestring \#a1e9a0...)]}.
$$

\noindent
No concrete syntax is provided for values of type
$\ty{bls12\_381\_mlresult}$. It is not possible to parse such values, and they
will appear as \texttt{(con bls12\_381\_mlresult <opaque>)} if output by a
program.


\note{Pairing operations.}
\label{note:pairing}
For efficiency reasons we split the pairing process into two parts:
the \texttt{bls12\_381\_millerLoop} and \texttt{bls12\_381\_finalVerify}
functions.  We assume that we have
\begin{itemize}
\item An intermediate multiplicative abelian group $H$.
\item A function (not necessarily itself a pairing) $e: G_1 \times
G_2 \rightarrow \MlResultDenotation$.
\item A cyclic group $\mu_r$ of order $r$.
\item An epimorphism $\psi: \MlResultDenotation \rightarrow \mu_r$ of groups such
that $\psi \circ e: G_1 \times G_2 \rightarrow \mu_r$ is a (nondegenerate,
bilinear) pairing.
\end{itemize}

\noindent Given these ingredients, we define
\begin{itemize}
\item $\denote{\TyMlResult} = \MlResultDenotation$.
\item $\denote{\mathtt{bls12\_381\_mulMlResult}} =$
the group multiplication operation in $\MlResultDenotation$.
\item $\denote{\mathtt{bls12\_381\_millerLoop}} = e$.
\item $\denote{\mathtt{bls12\_381\_finalVerify}} = \phi$,
where
$$
\phi(a,b) = \begin{cases}
               \true & \text{if $\psi(ab^{-1}) = 1_{\mu_r}$} \\
               \false & \text{otherwise.}
            \end{cases}
$$
\end{itemize}

\medskip
\noindent
We do not mandate specific choices for $\MlResultDenotation, \mu_r, e$, and $\phi$, but a
plausible choice would be
\begin{itemize}
\item $\MlResultDenotation = \units{\FF}$.
\item $e$ is the Miller loop associated with the optimal Ate pairing
for $E_1$ and $E_2$~\cite{Vercauteren}.
\item $\mu_r = \{x \in \units{\FF}: x^r=1\}$, the group of $r$th roots of unity in $\FF$.
(There are $r$ distinct $r$th roots of unity in $\FF$ because the embedding
degree of $E_1$ and $E_2$ with respect to $r$ is 12 (see~\cite[4.1]{Costello-pairings}).)
\item $\psi(x) = x^{\frac{q-1}{r}}$.
\end{itemize}

\noindent The functions \texttt{bls12\_381\_millerLoop} and (especially)
\texttt{bls12\_381\_finalVerify} are expected
to be expensive, so their use should be kept to a minimum.  Fortunately most
current use cases do not require many uses of these functions.



