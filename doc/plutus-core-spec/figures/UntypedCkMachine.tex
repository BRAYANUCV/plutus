\documentclass[../plutus-core-specification.tex]{subfiles}

\begin{document}


\begin{figure}[H]
\begin{subfigure}[c]{\linewidth}
    \centering
    \[\begin{array}{lrclr}
        \textrm{Stack} & s      & ::= & \cdot          & \textrm{empty stack}\\
                       &        &     & s,f            & \textrm{stack with frame $f$ at top}\\
        \textrm{State} & \sigma & ::= & s \compute M   & \textrm{computing a term}\\
                       &        &     & s \return V    & \textrm{returning a term value}\\
                       &        &     & \square V      & \textrm{halt and return a value}\\
                       &        &     & \ckerror{}     & \textrm{throw an error}
    \end{array}\]

    \captionof{figure}{Grammar of CK machine states for type-erased Plutus Core}
    \label{fig:untyped-ck-frames}
\end{subfigure}
\caption{A CK machine for type-erased Plutus Core}
\end{figure}

% Allow page break for (slightly) better placement
\begin{figure}[H]
\ContinuedFloat
  \begin{subfigure}[c]{\linewidth}
    \judgmentdef{$\sigma \mapsto \sigma^{\prime}$}{Machine takes one step from state $\sigma$ to state $\sigma'$}

%\hspace{-1cm}
    \begin{minipage}{\linewidth}
\begin{alignat*}{2}
  s & \compute \con{tn}{cn}                      &~\mapsto~& s \return \con{tn}{cn}\\
  s & \compute \lamU{x}{M}                       &~\mapsto~& s \return \lamU{x}{M}\\
  s & \compute \delay{M}                         &~\mapsto~& s \return \delay{M}\\
  s & \compute \force{M}                         &~\mapsto~& s,\inForceFrame{} \compute M \\
  s & \compute \appU{M}{N}                       &~\mapsto~& s,\inAppLeftFrame{N} \compute M\\
  s & \compute \builtinU{bn}{\!\!}               &~\mapsto~& s \compute M
                                                 \quad (\textit{$bn$ computes to $M$}) \\
  s & \compute \builtinU{bn}{M \repetition{M}}   &~\mapsto~& s,\inBuiltinU{bn}{}{\_}{\repetition{M}} \compute M\\
  s & \compute \errorU                           &~\mapsto~& \ckerror{}\\
  \\[-10pt] %% Put some vertical space between compute and return rules, but not a whole line
  \cdot & \return V                              &~\mapsto~& \square V\\
  s,\inAppLeftFrame{N} & \return V               &~\mapsto~& s,\inAppRightFrame{V} \compute N\\
  s,\inAppRightFrame{\lam{x}{A}{M}} & \return V  &~\mapsto~& s \compute \subst{V}{x}{M}\\
  s,\inForceFrame{} & \return \delay{M}          &~\mapsto~& s \compute M\\
  s,\inBuiltinU{bn}{\repetition{V}}{\_}{M \repetition{M}} & \return V
                                                 &~\mapsto~& s,\inBuiltinU{bn}{\repetition{V} V}{\_}{\repetition{M}}
                                                 \compute M\\
  s,\inBuiltinU{bn}{\repetition{V}}{\_}{} & \return V
                                                 &~\mapsto~&  s \compute M
                                                 \quad (\textit{$bn$ computes on $\repetition{V}V$ to $M$})\\
    \end{alignat*}
\end{minipage}
    \caption{CK machine transitions for type-erased Plutus Core}
    \label{fig:untyped-ck-transitions}
\end{subfigure}
\caption{A CK machine for type-erased Plutus Core}
\label{fig:untyped-ck-machine}
\end{figure}

%% The rule for builtins looks a bit strange. For one thing, it gives
%% us two ways to write built-in constants: $\con{X}$ and
%% $\builtinU{X}{\!\!}$.  I think this was deliberately included
%% in Rebecca's original typed syntax to give us a way to
%% instantiate polymorphic constants.  Roman says it will probably
%% still be useful for his extensible builtins.

\end{document}
