\section{Term Reduction}
\label{sec:reduction}

This section defines the semantics of (untyped) Plutus Core.

\subsection{Values in Plutus Core}
\label{sec:uplc-values}
The semantics of built-in functions in Plutus Core are obtained by instantiating
the sets $\Con{\tn}$ of constants of type $\tn$ (see
Section~\ref{sec:builtin-inputs}) to be the expressions of the form
\texttt{(con} $\tn$ $c$\texttt{)} and the set $\Inputs$ to be the set of Plutus
Core \textit{values}, terms which cannot immediately undergo any further
reduction, such as lambda terms and delayed terms.  Values also include partial
applications of built-in functions such as \texttt{[(builtin modInteger) (con
    integer 5)]}, which cannot perform any computation until a second integer
argument is supplied.  However, partial applications must also be
\textit{well-formed}, in the sense that applications of \texttt{force} must be
correctly interleaved with genuine arguments, and the arguments must themselves
be values.

We define syntactic classes $V$ of Plutus Core values and $A$ of partial builtin
applications simultaneously:

\begin{minipage}{\linewidth}
    \centering
    \[\begin{array}{lrcl}
        \textrm{Value}  & V   & ::= & \con{\tn}{c} \\
                        &     &     & \delay{M} \\
                        &     &     & \lamU{x}{M} \\
                        &     &     & \constr{i}{\repetition{V}} \\
                        &     &     & A
    \end{array}\]
    \captionof{figure}{Values in Plutus Core}
    \label{fig:untyped-cek-values}
\end{minipage}%
\nomenclature[F4]{$V$}{Plutus Core value}%
\nomenclature[F1]{$A$}{Well-formed partial built-in function application}

\medskip
\noindent Here $A$ is the class of well-formed partial applications, and to define
this we first define a class of possibly ill-formed iterated applications $\pba$ for
each built-in function $b \in \Fun$:

\begin{minipage}{\linewidth}
    \centering
  \[\begin{array}{lrl}
  \pba & ::= & \builtin{b}\\
       &     & \appU{\pba}{V}\\
       &    & \force{\pba}\\
    \end{array}\]
    \captionof{figure}{Partial built-in function application}
    \label{fig:partial-applications}
\end{minipage}%
\nomenclature[F2]{$\pba$}{Partial built-in function application (possibly ill-formed)}%
\nomenclature[F3]{$\pbas$}{Set of all  partial built-in function applications}

\medskip
\noindent We let $\pbas$ denote the set of terms generated by the grammar
in Figure~\ref{fig:partial-applications} and 
we define a function $\beta$ which extracts the name of the built-in
function occurring in a term in $\pbas$:
$$
 \begin{array}{ll}
 \beta(\builtin{b}) &= b\\
 \beta(\appU{\pba}{V}) & =\beta(\pba)\\
 \beta({\force{\pba}}) & =\beta(\pba)\\
\end{array}
$$%
\nomenclature[F5]{$\beta(\pba)$}{Function in partial builtin application $\pba$}

%% $$
%% \begin{array}{ll}
%%   \sat{\builtin{b}} &= []\\
%%   \sat{\appU{P}{V}} &= \sat{P}\snoc\type(V)\\
%%   \sat{\force{P}}   &= \sat{P}\snoc\fforce\\
%% \end{array}
%% $$

\noindent We also define a function $\pbasize{\cdot}$ which measures the size of
a term $\pba \in \pbas$:
$$
\begin{array}{ll}
\pbasize{\texttt{(builtin $b$)}} &= 0\\
\pbasize{\texttt{[$\pba$ $V$]}} &= 1+\pbasize{\pba}\\
\pbasize{\texttt{(force $\pba$)}} & = 1+\pbasize{\pba}
\end{array}
$$%
\nomenclature[F6]{$\pbasize{\pba}$}{Size of partial builtin application $\pba$}


%% \item Our built-in functions can take general members of $\Inputs$ as arguments
%%   as well as elements of the sets $\denote{\tn}$ and the symbol $\top$ is used
%%   to denote the type of elements of $\Inputs$. We use the symbol $\top$
%%   (which we assume does not appear in any other set we mention) to denote the
%%   type of non-constant elements of $\Inputs$ and write $\UniTop = \Uni \disj
%%   \{\top\}$ and $\UnihatTop = \Unihat \disj \{\top\}$.
%% \item We should be able to examine inputs (even during execution) to determine
%%   their types.  More precisely we assume that there is a function $\type:
%%   \Inputs \rightarrow \UniTop$ such that
%%   $$\type(x) =
%%   \begin{cases}
%%     \tn &\ \text{if } x \in \Con{\tn} \text{ for some } \tn \in \Uni\\
%%     \top &\text{otherwise}
%%   \end{cases}
%%   $$
%%   \noindent This is well defined because of our assumption that the sets $\Con{\tn}$ are disjoint.
%% \item We also define a partial order $\preceq$  on the set $\Uni^{\top}$ by
%%   $t_1 \preceq t_2$ if $t_1 = t_2$ or $t_2 = \top$.   


\paragraph{Well-formed partial applications.} A term $\pba \in \pbas$ is
an application of $b = \beta(\pba)$ to a number of values in $S$, interleaved
with applications of $\texttt{force}$.  We now define what it means for $\pba$
to be a \textit{well-formed partial application}.  Suppose that $\alpha(b) =
[\iota_1, \ldots, \iota_n]$. Firstly we require that $\pbasize{\pba} < n$, so
that $b$ is not fully applied; in this case we put
$\iota=\iota_{\pbasize{\pba}}$, the element of $b$'s signature which describes
what kind of ``argument'' $b$ currently expects.  The definition is completed by
induction on the structure of $\pba$:
\begin{enumerate}
\item $\pba=\mathtt{(builtin}\ b \mathtt{)}$ is always well-formed.
\item $\pba=\mathtt{[}\pba'\ V\mathtt{]}$ is well-formed if $\pba'$ is
  well-formed and $\iota \in \Unihash$ or $\iota \in \Var_*$ (equivalently, $\iota \notin \QVar$).
\item $\pba=\mathtt{(force}\ \pba'\mathtt{)}$ is well-formed if $\pba'$ is
  well-formed and $\iota \in \QVar$.
\end{enumerate}

\kwxm{Note that apart from type names all of this stuff is meta-notation that is
  need to describe the builtins machinery but isn't part of the language.}


\medskip
\noindent The definition of values in Figure~\ref{fig:untyped-cek-values} is now
completed by defining $A$ to be the syntactic class of well-formed
\textit{partial} built-in function applications:
$$
A = \{\pba \in \pbas: \pba \text{ is a well-formed partial application} \}.
$$

\noindent Note that this definition does not impose any requirements of type
correctness.  For example, with the types and functions defined in
Appendix~\ref{appendix:default-builtins-V1} the term $X =\texttt{[(builtin
    modInteger) (con string "blue")]}$ is a valid value which could be
treated like any other, for instance by being passed as an argument to a
\texttt{lam} expression.  However, the evaluation rules described in the next
section require that when a built-in function $b$ becomes \textit{fully} applied
the types of the arguments are checked against the signature of $b$ using the
relation $\approx$ and the function $\Eval$ defined in
Sections~\ref{sec:compatibility} and \ref{sec:eval}, so an error would arise if
the term $X$ were ever applied to another argument.



\paragraph{More notation.} Suppose that $A$ is a well-formed partial application with
$\alpha(\beta(A)) = [\iota_1,\ldots,\iota_n]$.  We define a function $\nextArg$
which extracts the next argument (or \texttt{force}) expected by $A$:
$$
    \nextArg(A) = \iota_{\pbasize{A}+1}.
$$
\noindent%
\nomenclature[Fr1]{$\nextArg(A)$}{Next argument type (or \texttt{force}) required by a partial builtin application $A$}
This makes sense because in a well-formed partial application $A$ we have
$\pbasize{A} < n$.

\medskip
\noindent We also define a function $\args{}$ which extracts the arguments which
$b$ has received so far in $A$:
$$
\begin{array}{ll}
  \args(\builtin{b}) &= []\\
  \args(\appU{A}{V}) &= \args(A)\snoc V\\
  \args(\force{A})   &= \args(A).\\
\end{array}
$$%
\nomenclature[Fr2]{$\args(A)$}{Term arguments received so far by partial builtin application $A$}

\subsection{Term reduction}

%% ---------------- Grammar of Reduction Frames ---------------- %%
\kwxm{Explain what this stuff means. Remember that when we apply the reduction
  rules we always use the first applicable one.

  I'm somewhat tempted to dump this in favour of SOS.}
\kwxm{Do we need a uniqueness condition on names somewhere?}
    
We define the semantics of Plutus Core using contextual semantics (or reduction
semantics): see~\cite{Felleisen-Hieb} or~\cite{Felleisen-Semantics-Engineering}
or~\cite[5.3]{Harper:PFPL}, for example.  We use $A$ to denote a partial
application of a built-in function as in Section~\ref{sec:uplc-values} above.
For builtin evaluation, we instantiate the set $\Inputs$ of
Section~\ref{sec:builtin-inputs} to be the set of Plutus Core values.  Thus all
builtins take values as arguments and return a value or $\errorX$.  Since values
are terms here, we can take $\reify{V} = V$.

\medskip
\noindent The notation $[V/x]M$ below denotes substitution of the value $V$ for
the variable $x$ in $M$.  This is \textit{capture-avoiding} in that substitution
is not performed on occurrences of $x$ inside subterms of $M$ of the form
$\lamU{x}{N}$.%
\nomenclature[Fr3]{$[V/x]M$}{Capture-avoiding substitution of value $V$ for variable $x$ in term $M$}

\begin{figure}[H]
\begin{subfigure}[c]{\linewidth}
    \centering
    \[\begin{array}{lrclr}
        \textrm{Frame} & f  & ::=   & \inAppLeftFrame{M}                                       & \textrm{left application}\\
                       &   &     & \inAppRightFrame{V}                                         & \textrm{right application}\\
                       &   &     & \inForceFrame                                               & \textrm{force}\\
                       &   &     & \inConstrFrame{i}{\repetition{V}}{\repetition{M}}           & \textrm{constructor argument}\\
                       &   &     & \inCaseFrame{\repetition{M}}                                & \textrm{case scrutinee}
    \end{array}\]
    \caption{Grammar of reduction frames for Plutus Core}
    \label{fig:untyped-reduction-frames}
\end{subfigure}
%\end{figure}%
\nomenclature[Fr3]{$f$}{Reduction frame for contextual semantics: $\inAppLeftFrame{M}, \inAppRightFrame{V}, \inForceFrame$}



\bigskip
%\begin{figure}[H]
%\ContinuedFloat
%% ---------------- Reduction via Contextual Semantics ---------------- %%
\begin{subfigure}[c]{\linewidth}
  % \def\labelSpacing{20pt}

  \judgmentdef{$\step{M}{M'}$}{Term $M$ reduces in one step to term $M'$.}

   % [(lam x M) V] -> [V/x]M
    \begin{prooftree}
        \AxiomC{}
        % \RightLabel{\textsf{apply-lambda}}
        \UnaryInfC{$\step{\app{\lamU{x}{M}}{V}}{\subst{V}{x}{M}}$}
    \end{prooftree}

    % [A V] saturated
    \begin{prooftree}
      \AxiomC{$\length(A) = \chi(\beta(A))-1$}
      \AxiomC{$\nextArg(A) \in \Unihash \cup \Var_*$}
        % \RightLabel{\textsf{final-apply}}
        \BinaryInfC{$\step{\app{A}{V}}{\Eval'(\beta(A), \args(A)\snoc V)}$}
    \end{prooftree}

    % [A V] unsaturated
    \begin{prooftree}
      \AxiomC{$\length(A) < \chi(\beta(A))-1$}
      \AxiomC{$\nextArg(A) \in \Unihash \cup \Var_*$}
        % \RightLabel{\textsf{intermediate-apply}}
        \BinaryInfC{$\step{\app{A}{V}}{\app{A}{V}}$}
    \end{prooftree}

    % force (delay M) -> M
    \begin{prooftree}
        \AxiomC{}
        % \RightLabel{\textsf{force-delay}}
        \UnaryInfC{$\step{\force{\delay{M}}}{M}$}
    \end{prooftree}

    % case (constr i vs) cs -> [c_i vs]
    \begin{prooftree}
        \AxiomC{$0 \leq i \leq m$}
        % \RightLabel{\textsf{case-of-constr}}
        \UnaryInfC{$\step{\kase{\constr{i}{\repetition{V}}}{U_0 \ldots U_m}}{\app{U_i}{\repetition{V}}}$}
    \end{prooftree}

    % Saturated force
    \begin{prooftree}
      \AxiomC{$\length(A) = \chi(\beta(A))-1$}
      \AxiomC{$\nextArg(A) \in \QVar$}
        % \RightLabel{\textsf{final-force}}
        \BinaryInfC{$\step{\force{A}}{\Eval'(\beta(A), \args(A))}$}
    \end{prooftree}


    % Unsaturated force
    \begin{prooftree}
      \AxiomC{$\length(A) < \chi(\beta(A))-1$}
      \AxiomC{$\nextArg(A) \in \QVar$}
        % \RightLabel{\textsf{intermediate-force}}
        \BinaryInfC{$\step{\force{A}}{A}$}
    \end{prooftree}

%    \hfill\begin{minipage}{0.3\linewidth}  
      \begin{prooftree}
        \AxiomC{} % If we're putting these side by side we need \strut here to get rules aligned
        % \RightLabel{\textsf{error}}
        \UnaryInfC{$\step{\ctxsubst{f}{\errorU}}{\errorU}$}
      \end{prooftree}
%    \end{minipage}
%    \begin{minipage}{0.3\linewidth}
    \begin{prooftree}
        \AxiomC{$\step{M}{M'}$}  % Need \strut for side-by-side alignment again
        \UnaryInfC{$\step{\ctxsubst{f}{M}}{\ctxsubst{f}{M'}}$}
    \end{prooftree}
% \end{minipage}\hfill\hfill %% Don't know why we need two \hfills here but only one at the start
% \\
    \caption{Reduction via Contextual Semantics} %% Oops
    \label{fig:untyped-reduction}
\end{subfigure}

\bigskip

\begin{subfigure}[c]{\linewidth}
  $$ \Eval'(b, [V_1, \ldots, V_n]) =
  \begin{cases}
  \errorU  & \text{if $\Eval(b,[V_1, \ldots, V_n]) = \errorX$}\\
  \Eval(b,[V_1, \ldots, V_n]) & \text{otherwise}
  \end{cases}
$$
    \caption{Built-in function application}
    \label{fig:bif-appl}
\end{subfigure}

\caption{Term reduction for Plutus Core}
\label{fig:untyped-term-reduction}
\end{figure}

\bigskip
\noindent It can be shown that any closed Plutus Core term whose evaluation
terminates yields either \texttt{(error)} or a value. Recall from
Section~\ref{sec:grammar-notes} that we require the body of every Plutus Core
program to be closed.

\kwxm{I was worried because we only have rules for eg application of a builtin
  $b$ to a final argument, and when we're applying $b$ to other arguments we
  don't check that a term argument is actually expected (rather than a
  \texttt{force}), and that the argument has the right type.  I think this is OK
  though: for example, if we have a builtin $b$ with $\arity{b} = [\forall a_\#,
    \texttt{int}]$ and we have a term $M = \texttt{[(builtin b) (con 5)]}$ then
  none of the rules apply because $M$ isn't in $A$, so the semantics get stuck.
  This happens in general as well: the definition of $A$ doesn't even let us
  talk about partial builtin applications where the interleaving is wrong.  We
  \textit{do} need special rules for the final argument because if $M \in A$ we
  have to look at $b$ to make sure that the final argument (or force) is the
  right kind of thing.}
  
  
