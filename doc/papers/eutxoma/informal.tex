\section{Extending Extended \UTXO}
\label{sec:EUTXOma}

Before discussing applications and the formal model of \EUTXOma, we briefly summarise the structure of \EUTXO, and then informally introduce the multi-asset extension that is the subject of this paper.
Finally, we will discuss a shortcoming in the state machine mapping of \EUTXO\ as introduced in~\cite{eutxo-1-paper} and illustrate how the multi-asset extension fixes that shortcoming.

\subsection{The starting point: Extended \UTXO}
\label{sec:informal-eutxo}

In Bitcoin's \UTXO\ ledger model, the ledger is formed by a list of transactions grouped into blocks.
As the block structure is secondary to the discussion in this paper, we will disregard it in the following.
A transaction $\tx$ is a quadruple \((I, O, r, S)\) comprising a set of \emph{inputs} $I$, a list of \emph{outputs} $O$, a \emph{validity interval} $r$, and a set of signatures $S$, where inputs and outputs represent cryptocurrency value flowing into and out of the transaction, respectively.
The sum of the inputs must be equal to the sum of the outputs; in other words, transactions preserve value.
% True, but not crucial IMHO and we need to save space. -= chak
%\footnote{
%  Real transactions will likely have other fields that need to be balanced, such as fees or new tokens that are minted or forged.
%  We will omit these for now for simplicity, but in general they also need to observe preservation of value.
%}
Transactions are identified by a collision-resistant cryptographic hash $h$ computed from the transaction.\footnote{%
  We denote the hash of some data $d$ as $d^\#$.%
}

An input $i\in I$ is represented as a pair \((\outRef, \rho)\) of an \emph{output reference} $\outRef$ and a \emph{redeemer value} $\rho$.
The output reference $\outRef$ uniquely identifies an output in a preceding transaction by way of the transaction's hash and the output's position in the transaction's list of outputs.

In plain \UTXO, an output $o\in O$ is a pair of a \emph{validator script} $\nu$ and cryptocurrency value $\val$.
In the Extended \UTXO\ model (\EUTXO)~\cite{eutxo-1-paper}, outputs become triples \((\nu, \val, \delta)\), where the added \emph{datum} $\delta$ enables passing additional information to the validator.

The purpose of the validator is to assess whether an input $i$ of a subsequent transaction trying to spend (i.e., consume) an output $o$ should be allowed to do so.
To this end, we execute the validator script to check whether \(\nu(\rho, \delta, \sigma) = \true\) holds.
Here $\sigma$ comprises additional information about the \emph{validation context} of the transaction.
In the plain \UTXO\ model that contextual information is fairly limited: it mainly consists of the validated transaction's hash, the signatures $S$, and information about the length of the blockchain. In the \EUTXO\ model, we extend $\sigma$ to include the entirety of the validated transaction $\tx$ as well as all the outputs spent by the inputs of $\tx$.

\subsection{Token bundles}
\label{sec:informal-token-bundles}

In \UTXO\ and \EUTXO, the $\val$ carried by an output is represented as an integral value denoting a specific quantity of the ledger's native cryptocurrency.
As discussed in more detail in the companion paper~\cite{plain-multicurrency}, we can generalise $\val$ to carry a two-level structure of \emph{finitely-supported functions.}
The technicalities are in Appendix~\ref{app:model}; for now, we can regard them as nested finite maps to quantities of tokens.
For example, the value
\(\{\mathsf{Coin} \mapsto \{\mathsf{Coin} \mapsto 3\}, g \mapsto \{t_1
\mapsto 1, t_2 \mapsto 1\}\}\) contains $3$ $\mathsf{Coin}$ coins
(there is only one (fungible) token $\mathsf{Coin}$ for a payment
currency also called $\mathsf{Coin}$), as well as (non-fungible) tokens $t_1$ and $t_2$,
both in asset group $g$.  Values can be added naturally, e.g.,
\begin{align*}
 & \{\mathsf{Coin} \mapsto \{\mathsf{Coin} \mapsto 3\}, g \mapsto \{t_1
   \mapsto 1, t_2 \mapsto 1\}\} \\
 + \ & \{\mathsf{Coin} \mapsto \{\mathsf{Coin} \mapsto 1\}, g \mapsto \{t_3 \mapsto 1\}\} \\
 = \ & \{\mathsf{Coin} \mapsto \{\mathsf{Coin} \mapsto 4\}, g \mapsto \{t_1
       \mapsto 1, t_2 \mapsto 1, t_3 \mapsto 1\}\} \ .
\end{align*}
%
In a bundle, such as \(g \mapsto \{t_1 \mapsto 1, t_2 \mapsto 1, t_3 \mapsto 1\}\), we call $g$ an \emph{asset group} comprising a set of tokens $t_1$, $t_2$, and $t_3$. In the case of a fungible asset group, such as $\mathsf{Coin}$, we may call it a \emph{currency}.
In contrast to fungible tokens, only a single instance of a non-fungible token may be minted.

\subsection{Forging custom tokens}

To enable the introduction of new quantities of new tokens on the ledger (\emph{minting}) or the removal of existing tokens (\emph{burning}), we add a \emph{forge field} to each transaction.
The use of the forge field needs to be tightly controlled, so that the minting and burning of tokens is guaranteed to proceed according to the token's \emph{forging policy}.
We implement forging policies by means of scripts that are much like the validator scripts used to lock outputs in \EUTXO.

Overall, a transaction in \EUTXOma\ is thus a sextuple \((I, O, r, \forge, \fpss, S)\), where $\forge$, just like $\val$ in an output, is a token bundle and $\fpss$ is a set of \emph{forging policy scripts}.
Unlike the value attached to transaction outputs, $\forge$ is allowed to contain negative quantities of tokens. Positive quantities represent minted tokens and negative quantities represent burned tokens.
In either case, any asset group $\phi$ that occurs in $\forge$ (i.e., \(\forge = \{\ldots, \phi \mapsto \mathit{toks}, \ldots\}\)) must also have its forging policy script in the transaction's $\fpss$ field.
Each script $\pi$ in $\fpss$ is executed to check whether $\pi(\sigma) = \true$, that is whether the transaction, including its $\forge$ field, is admissible.
\ifootnote{%
  In fact we have a slightly different $\sigma$ here, which we elaborate on in Appendix~\ref{app:model}.%
}

\subsection{Constraint emitting machines}
\label{sec:cem-example}
In the \EUTXO\ paper~\cite{eutxo-1-paper}, we explained how we can map \emph{Constraint Emitting Machines} (CEMs) ---a variation on Mealy machines--- onto an \EUTXO\ ledger. A CEM consists of its type of states \s{S} and inputs
\s{I}, predicates $\s{initial}, \s{final} : \s{S} \rightarrow \B$
indicating which states are initial and final, respectively, and a valid set of transitions,
given as a partial function $\s{step} : \s{S} \rightarrow \s{I} \rightarrow
\s{Maybe}\ (\s{S} \times \s{TxConstraints})$ from source state and input symbol to
target state and constraints.
\ifootnote{%
  The result may be \s{Nothing}, in case no valid transitions exist from a given state/input.%
}
%
One could present CEMs using the traditional five-tuple notation $(Q, \Sigma, \delta, q_0, F)$,
but we opt for functional notation to avoid confusion with standard \textit{finite state machines}
(see \cite{eutxo-1-paper} on how CEMs differ from FSMs),
as well as being consistent with our other definitions.

A sequence of CEM state transitions, each of the form $\CStepArrow{s}{i}{\txeq}$, is mapped to a sequence of transactions, where each machine state $s$ is represented by one transaction $\tx_s$. Each such transaction contains a \emph{state machine output} $o_s$ whose validator $\nu_s$ implements the CEM transition relation and whose datum $\delta_s$ encodes the CEM state $s$.

The transition $\tx_{s'}$, representing the successor state, spends $o_s$ with an input that provides the CEM input $i$ as its redeemer $\rho_i$. Finally, the constraints $\txeq$ generated by the state transition need to be met by the successor transition $\tx_{s'}$. (We will define the correspondence precisely in \S\ref{sec:formalization}.)

\begin{figure}[t]
  \centering
  \begin{tikzpicture}[
    state/.style =
    { shape        = circle,
      fill         = gray!40,
      align        = center,
      font         = \large,
      minimum size = 2.5cm },
    transition/.style =
    { ->,
      thick },
    move/.style =
    { align  = center,
      anchor = center,
      font   = \small }
  ]
  \node[state] (a) {\Holding};
  \node[state, right = 3cm of a] (b) {\msf{Collecting}\\\normalsize ($v$, $p$, $d$, $sigs$)};

  \path
  (a.north east) edge[transition, bend left = 15]
    node[move, above] {$\Propose{v}{p}{d}$\\$sigs' = \{\}$}
  (b.north west)
  (b) edge[transition, out = 30, in = -30, looseness = 3]
      node[move, right] {\Add{sig}\\$sigs' = sigs \cup sig$\\ if $sig \in sigs_{auth}$}
      (b)
      edge[transition, bend left = 15]
      node[move, above] {\Pay\\if $|sigs| \geq n$}
      (a)
      edge[transition, bend left = 30]
      node[move, below] {\Cancel\\if $d$ expired}
      (a)
  ;
  \end{tikzpicture}
  \caption{Transition diagram for the multi-signature state machine;
    edges labelled with input from redeemer and transition constraints.}
  \label{fig:multisig-machine}
\end{figure}

A simple example for such a state machine is an on-chain $n$--of--$m$ multi-signature contract. Specifically, we have a given amount
$\val_\msc$ of some cryptocurrency and we require the approval of at
least $n$ out of an \textit{a priori} fixed set of $m \geq n$ owners to spend
$\val_\msc$. With plain \UTXO{} (e.g., on Bitcoin), a multi-signature
scheme requires out-of-band (off-chain) communication to collect all
$n$ signatures to spend $\val_\msc$. On Ethereum, and also in the
\EUTXO{} model, we can collect the signatures on-chain, without any
out-of-band communication. To do so, we use a state machine operating
according to the transition diagram in
Figure~\ref{fig:multisig-machine}, where we assume that the threshold
$n$ and authorised signatures $\sigs_\auth$ with \(|\sigs_\auth| = m\)
are baked into the contract code.

In the multi-signature state machine's implementation in the \EUTXO{} model, we use a validator
function $\nu_\msc$ accompanied by the datum $\delta_\msc$ to lock
$\val_\msc$. The datum $\delta_\msc$ stores the machine state,
which is of the form \(\Holding\) when only holding the locked value
or \(\Collecting{\val}{\kappa}{d}{\sigs}\) when collecting
signatures $\sigs$ for a payment of $\val$ to $\kappa$ by the deadline
$d$. The initial output for the contract is \((\nu_\msc, \val_\msc,
\Holding)\).

The validator $\nu_\msc$ implements the state transition diagram from
Figure~\ref{fig:multisig-machine} by using the redeemer of the spending input to determine the transition that needs to be taken. That redeemer (state machine input) can take four forms:
\begin{inparaenum}[(1)]
\item \(\Propose{\val}{\kappa}{d}\) to propose a payment of $\val$ to $\kappa$
  by the deadline $d$,
\item \(\Add{\sig}\) to add a signature $\sig$ to a payment,
\item $\Cancel$ to cancel a proposal after its deadline expired, and
\item $\Pay$ to make a payment once all required signatures have been collected.
\end{inparaenum}
It then validates that the spending transaction $\mi{tx}$ is a valid
representation of the newly reached machine state. This implies that
$\mi{tx}$ needs to keep $\val_\msc$ locked by $\nu_\msc$ and that the
state in the datum $\delta^{\prime}_\msc$ needs to be the successor state
of $\delta_\msc$ according to the transition diagram.

While the state machine in Figure~\ref{fig:multisig-machine} is fine, its mapping to \EUTXO\ transactions comes with a subtle caveat: what if, for a 2--of--3 contract, somebody posts a transition $\tx_{\mathrm{bad}}$ corresponding to the state \(\Collecting{\val}{\kappa}{d}{\{\sig_1, \sig_2\}}\) onto the chain \emph{without} going through any of the previous states, including without starting in $\Holding$?
Given that $\Pay$ merely checks \(|\sigs| \geq 2\), the payment would be processed when requested on $\tx_{\mathrm{bad}}$, even if \(\{\sig_1, \sig_2\}\) are invalid signatures.
We would not have been allowed to add the invalid signatures $\sig_1$ and $\sig_2$ by way of \(\Add{\sig}\) since its state transition checks signature validity, but by initialising the state machine in an intermediate state $s$ with \(\s{initial}(s) = \false\), we were able to circumvent that check.

In the plain \EUTXO\ model, there is no simple way for the validator implementing a state machine to assert that the state it is operating on arose out of a succession of predecessor states rooted in an initial state.
As a consequence of this limitation, the formal result presented in the \EUTXO\ paper~\cite{eutxo-1-paper} is not as strong as one might hope.
More precisely, this previous work did establish soundness and completeness for a weak bisimulation between CEMs and transactions on a \EUTXO\ ledger;
however, it fell short in that it did not show that an inductive property met by the states of a CEM may also be asserted for the corresponding states on the ledger.
The reason for this omission is precisely the problem we just discussed: for a ledger-representation of a CEM state, we can never be sure that it arose out of a transaction sequence rooted in an initial CEM state.

\subsection{Token provenance}
\label{sec:state-machine-provenance}

In the present work, we solve the problem described above. We will show that non-fungible tokens can be used to identify a unique trace of transactions through an \EUTXOma\ ledger that corresponds to all the states of the corresponding CEM. Moreover, we introduce a notion of \emph{token provenance} that enables us to identify the first CEM state in such a trace and to ensure that it is a state accepted by \(\s{initial} : \s{S} \rightarrow \B\). Together, these allow a state machine validator to ensure that a current CEM state indeed arose out of a trace of transactions rooted in a transaction corresponding to an initial CEM state merely by checking that the non-fungible token associated with this state machine instance is contained in the $\val$ of the state machine output that it locks: this would be $\val_\msc$ for the multi-signature contract of \S\ref{sec:cem-example}.
