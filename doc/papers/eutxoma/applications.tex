\section{Applications}
\label{sec:applications}

The functionality and applications discussed in the companion paper~\cite{plain-multicurrency} on \UTXOma\ are all still possible in the extended \EUTXOma\ model discussed here. In fact, some of the applications can even be realised more easily by using the added \EUTXO\ functionality, for example by using a state machine to control the behaviour of a forging policy over time. On top of that, we can realise new applications and move some of the functionality that needed to be implemented by an off-chain trusted party in \UTXOma\ into on-chain script code in \EUTXOma.

\subsection{State thread tokens}

As discussed in \S\ref{sec:EUTXOma}, many interesting smart contracts can be implemented by way of state machines. However, without custom tokens, the implementation of state machines on an \EUTXO\ ledger suffers from two problems:
%
\begin{itemize}
\item Multiple instances of a single state machine (using the same validator code) generate state machine outputs that look alike. Hence, for a given output locked by the state machine's validator, we cannot tell whether it belongs to one or another run of that state machine.
\item The issue raised in \S\ref{sec:cem-example}: a validator cannot tell whether the state machine was started in one of its initial states or whether somebody produced out of thin air a state machine in the middle of its execution.
\end{itemize}
%
We can work around the first problem by requiring all transactions of a particular state machine instance to be signed by a particular key and checking that as part of state machine execution. This is awkward, as it requires off-chain communication of the key  for a multi-party state machine. In any case these two problems open contracts up to abuse if they are not carefully implemented.

We can solve both of these problems with the help of a unique
non-fungible token, called a \emph{state thread token,} which
is minted in an initial CEM state. From the initial state on, it uniquely identifies the transaction
sequence implementing the progression of one specific instance of the token's
state machine.
The state thread token is passed from one transition to the next by
way of the state machine output of each transaction, where the
presence of the state thread token is asserted by the state machine
validator. More precisely, assuming a validator $\nu_s$ implementing
the state machine and a matching forging policy $\phi_s$, the
state thread token of the $\phi_s$ policy is used as follows:
%
\begin{itemize}

\item The forging policy $\phi_s$ checks that
\begin{itemize}
\item the transaction mints a unique, non-fungible token $\mathit{tok}_s$ and
\item the transaction's state machine output is locked by the state machine validator $\nu_s$ and contains an admissible initial state of the state machine in its datum field.
\end{itemize}

\item The state machine validator $\nu_s$, in turn, asserts that
\begin{itemize}
\item the standard state machine constraints hold,
\item the value locked by $\nu_s$ contains the state thread token $\mathit{tok}_s$, and
\item if the spending transaction represents a final state, it burns $\mathit{tok}_s$.
(We ignore burning in the formalisation in \S\ref{sec:formalization}. It is for cleaning up and not relevant for correctness.)
\end{itemize}
\end{itemize}

This solves both problems. The uniqueness of the non-fungible token ensures that there is a single path that represents the progression of the state machine. Moreover, as the token can only be minted in an initial state, it is guaranteed that the state machine must indeed have begun in an initial state. In \S\ref{sec:formalization} we formalise this approach and we show that it is sufficient to give us the properties that we want for state machines.

\subsection{Tokenised roles and contracts}

Custom tokens are convenient to control ownership and access to smart contracts. In particular, we can turn \emph{roles} in a contract into tokens by forging a set of non-fungible tokens, one for each role in the contract. In order to take an action (for example, to make a specific state transition in a state machine) as that role, a user must present the corresponding \emph{role token}.

For example, in a typical financial futures contract we have two roles: the buyer (long), and the seller (short).
In a tokenised future contract, we forge tokens which provide the right to act as the long or short position; e.g., in order to settle the future and take delivery of the underlying asset, an agent needs to provide the long token as part of that transaction.

\paragraph{Trading role tokens.}
%
Tokenised roles are themselves \emph{resources} on the ledger, and as such are tradeable and cannot be split or double spent,
which is useful in practice.
For example, it is fairly typical to trade in futures contracts, buying or selling the right to act as the long or short position in the trade.
With tokenised roles, this simply amounts to trading the corresponding role token.

The alternative approach is to identify roles with public keys, where users \emph{authenticate} as a role by signing the transaction with the corresponding key.
However, this makes trading roles much more cumbersome than simply trading a token;
an association between roles and keys could be stored in the contract datum, but this requires interacting with the contract to change the association.
Moreover, when using public keys instead of role tokens, we cannot easily implement more advanced use cases that require treating the role as a resource, such as the derivatives discussed below.

The lightweight nature of our tokens is key here.
To ensure that roles for all instances of all contracts are distinct, every instance  requires a unique asset group with its own forging policy.
If we had a global register with all asset groups and token names, adding a new asset group and token for every set of role tokens we create would add significant overhead.

\paragraph{Derivatives and securitisation.}
%
Since tokenised roles are just tokens, they are themselves assets that can be managed by smart contracts.
This enables a number of derivative (higher-order) financial contracts to be written in a generic way, without requiring a hard-coded list of tokens as the underlying assets.

For example, consider $\mathsf{interest}$, a contract from which payments can be extracted at regular intervals, based on some interest rate.
We can tokenise this by issuing a token for the \emph{creditor} role that represents the claim to those payments: if someone wishes to claim the payment then they must present that token.

If we have several instances of $\mathsf{interest}$ (perhaps based on different interest rates), we can lock their creditor tokens in a new contract that bundles the cash flows of all the underlying contracts.
The payments of this new contract are the sum of the payments of the $\mathsf{interest}$ contracts that it collects.
The tokens of the $\mathsf{interest}$ contracts cannot be traded separately as long as they are locked by the new contract.

This kind of bundling of cash flows is called \emph{securitisation}.
Securitisation is commonly used to even out variations in the payment streams of the underlying contracts and to distribute their default risk across different risk groups (tranches).
Derivative contracts, including the example above, make it possible to package and trade financial risks, ultimately resulting in lower expenditure and higher liquidity for all market participants.
Our system makes the cost of creating derivatives very low, indeed no higher than making a contract that operates on any other asset.

\subsection{Fairness}

Tokens can be used to ensure that \emph{all} participants in some agreement have been involved. For example, consider a typical ICO setup in which a number of participants pay for the right to buy the token during an initial issuance phase.
A naive implementation is as follows:
%
\begin{itemize}
\item
  Contributors lock their contributions with a validator $\nu$ and a datum $\delta$, where $\delta$ contains their public key and $\nu$ requires that an appropriate part of the forged tranche be sent to the public key address corresponding to $\delta$.
\item
  The forging policy $\phi$ requires that the sum of the inputs exceeds a pre-determined  threshold $T$, and allows forging of $n$ units of the token, which must be allocated to the input payers in proportion to their contribution.
\end{itemize}

Unfortunately, this is not \emph{fair}, in that participants can be omitted by the party who actually creates the forging transaction, as long as the other participants between them reach the threshold.

We can fix this by issuing (fungible) \emph{participation tokens}, representing the right to participate in the ICO.
First, we forge $l$ participation tokens, and then we distribute them to the participants (in return for whatever form of payment we require).
Finally, in order to issue the main tranche of tokens, we require (in addition to the previous conditions) that some appropriate fraction of the issued participation tokens are spent in that transaction.
That means we cannot omit (too many of) the holders of the participation tokens --- and the forging policy ensures that all participants are compensated appropriately.
As a bonus, this makes the right to participate in the ICO itself tradeable as a tokenised role. In other words, participation tokens make roles into \emph{first-class} assets.

A similar scheme is being used in the Hydra head protocol~\cite{chakravarty2020hydra}, which implements a layer-2 scalability solution. \emph{Hydra heads} enable groups of participants to advance an a priori locked portion of the mainchain \UTXO\ set in a fast off-chain protocol with fast settlement in a manner that provides the same level of security as the mainchain. The mainchain portion of the protocol, which is based on \EUTXOma, uses custom tokens to ensure that it is impossible for a subgroup of the participants to collude to exclude one or more other participants.

\subsection{Algorithmic stablecoins}
\label{sec:stablecoins}

In \cite{plain-multicurrency}, we described how to implement a simple, centralised stablecoin in the \UTXOma{}.
However, more sophisticated stablecoin designs exist, such as the Dai of MakerDAO~\cite{team2017dai}.

Stablecoins where more of the critical functionality is validated on-chain can be realised within the \EUTXOma{} model. For example, we can use a state machine that acts as a \emph{market maker} for a stablecoin by forging stablecoins in exchange for other assets. Mechanisms to audit updates to the current market price,  or to suspend trading if the fund's liabilities become too great, can also be implemented programmatically on-chain.
